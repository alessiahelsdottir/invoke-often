\addchap{Liber RV vel Spiritus}\index{Pranayama}\index{Grade Studies \& Work!Zelator (\grade{2}{9})}\index{Grade Studies \& Work!Probationer (\grade{0}{0})}\index{Grade Studies \& Work!Zelator (\grade{2}{9})}
\chapnum{CCVI\footnote{The number of R V, referred to in the text. From the Equinox Vol. 1, No. 7; Magick in Theory and Practice.}}

\textbf{Full instruction in Pranayama.} \editorsnote{There was no line 1 in either original.}

2. Let the Zelator observe the current of his breath.

3. Let him investigate the following statements, and prepare a careful record\index{Recording results} of research.

\begin{enumerate}[label=(\textit{\alph*})]
\item Certain actions induce the flow of the breath through the right nostril (Pingala); and, conversely, the flow of the breath through Pingala induces certain actions.
\item Certain other actions induce the flow of the breath through the left nostril (Ida), and conversely.
\item Yet a third class of actions induce the flow of the breath through both nostrils at once (Sushumna), and conversely.
\item The degree of mental and physical activity is interdependent with the distance from the nostrils at which the breath can be felt by the back of the hand.
\end{enumerate}

4. \textit{First practice.} Let him concentrate his mind on the act of breathing, saying mentally, \enquote{the breath flows in}, \enquote{the breath flows out}, and record the results. (This practice may resolve itself into Mahasatipatthana (\textit{vide} Liber XXV)\editorsnote{What this is referring to is unclear. This predates the publication of the Star Ruby as XXV. There's another reference to it in Liber Viarum Vi\ae{}, from the same Equinox issue, as part of the grade of \cjRL{n|}{}, \enquote{The Preparation of the Corpse for the Tomb}; also unclear.} or induce Samadhi. Whichever occurs should be followed up as right Ingenium of the Zelator, or the advice of his Practicus, may determine.)

5. \textit{Second practice.} Pranayama. This is outlined in Liber E. Further, let the Zelator accomplished in those practices endeavour to master a cycle of 10, 20, 40 or even 16, 32, 64. But let this be done gradually and with due caution. And when he is steady and easy both in Asana and Pranayama, let him still further increase the period.

Thus let him investigate these statements which follow:
\begin{enumerate}[label=(\textit{\alph*})]
\item If Pranayama be properly performed, the body will first of all become covered with sweat. This sweat is different in character from that customarily induced by exertion. If the Practitioner rub this sweat thoroughly into his body, he will greatly strengthen it.
\item The tendency to perspiration will stop as the practice is continued, and the body become automatically rigid.
\end{enumerate}

Describe this rigidity with minute accuracy.

\begin{enumerate}
\item[(\textit{c})] The state of automatic rigidity\index{Automatic rigidity} will develop into a state characterised by violent spasmodic movements of which the Practitioner is unconscious, but of whose result he is aware. This result is that the body hops gently from place to place. After the first two or three occurrences of this experience, Asana is not lost. The body appears (on another theory) to have lost its weight almost completely and to be moved by an unknown force.\editorsnote{In the Equinox Vol. 1, No. 7. p. 61, available online today, there are a set of images of Aleister Crowley performing this operation. They are useful to review, but have been omitted for space.}
\item[(\textit{d})] As a development of this stage, the body rises into the air, and remains there for an appreciably long period, from a second to an hour or more.
\end{enumerate}
Let him further investigate any mental results which may occur.

6. \textit{Third practice.} In order both to economise his time and to develop his powers, let the Zelator practice the deep full breathing which his preliminary exercises will have taught him during his walks. Let him repeat a sacred sentence\index{Sacred sentence|see{Mantras}} (mantra\index{Mantras!in pranayama}) or let him count, in such a way that his footfall beats accurately with the rhythm thereof, as is done in dancing. Then let him practice Pranayama, at first without the Kumbhakam\index{Kumbhaka},\editorsnote{\enquote{In all books they speak of Pranayama being divided into Rechaka (rejecting or exhaling), Puraka (inhaling), Kumbhaka (restraining, stationary).} -- Raja Yoga, Swami Vivekananda. Internal kumbhaka (inhale, hold) is known as \enquote{Antara} Kumbhaka; external Kumbhaka (exhale, hold) is known as \enquote{Bahya Kumbhaka}.} and paying no attention to the nostrils or otherwise than to keep them clear. Let him begin by an indrawing of breathing for 4 paces, and a breathing out for 4 paces. Let him increase this gradually to 6.6, 8.8, 12.12, 16.16, and 24.24, or more if he be able. Next let him practice in the proper proportion 4.8, 6.12, 8.16, 12.24, and so on. Then if choose, let him recommence the series, adding a gradually increasing period of Kumbhakam.

7. \textit{Fourth practice.} Following on this third practice, let him quicken his mantra and his pace until the walk develops into a dance. This may also be practiced with the ordinary waltz step, using a mantra in three-time, such as \textgreek{\'{ε}πελθον}, \textgreek{\'{ε}πελθον}, \textgreek{Αρτεμις}; or \textsc{\textgreek{Ιαο}; \textgreek{Ιαο Σαβαο}}\index{Formul\ae{}!\textgreek{ΙΑΟ}}; in such cases the practice may be combined with devotion to a particular deity: see Liber CLXXV\editorsnote{Liber Astarte vel Berylli}. For the dance as such it is better to use a mantra of a non-committal character, such as \textgreek{το} \textgreek{ειναι}, \textgreek{το} \textgreek{καλον}, \textgreek{το’γαθον}\editorsnote{There are a couple variations in these examples in different publications. Remember that these are \textit{examples}, not the only options. Find one that suits you.}, or the like.

8. \textit{Fifth practice.} Let him practice mental concentration\index{Dharana}\editorsnote{\textit{I.e.}, dharana (as in Liber E) on the dance and mantra.} during the dance, and investigate the following experiments:

\begin{enumerate}[label=(\alph*)]
\item The dance becomes independent of the will.
\item Similar phenomena to those described in 5 (a), (b), (c), (d), occur.
\item Certain important mental results occur.
\end{enumerate}

9. A note concerning the depth and fulness of the breathing. In all proper expiration the last possible portion of air should be expelled. In this the muscles of the throat, chest, ribs, and abdomen must be fully employed, and aided by the pressing of the upper arms into the flanks, and of the head into the thorax.

In all proper inspiration the last possible portion of air must be drawn into the lungs.

In all proper holding of the breath, the body must remain absolutely still.

Ten minutes of such practice is ample to induce profuse sweating in every place of a temperature $17^{\circ}C.$ or over.

The progress of the Zelator in acquiring a depth and fulness of breath should be tested by the respirometer.

The exercises should be carefully graduated to avoid overstrain and possible damage to the lungs.

This depth and fulness of breath should be kept as much as possible, even in rapid exercises, with the exception of the sixth practice following.

10. \textit{Sixth practice.} Let the Zelator breathe as shallowly and rapidly as possible. He should assume the attitude of his moment of greatest expiration, and breathe only with the muscles of his throat. He may also practice lengthening the period between each shallow breathing.

(This may be combined, when acquired, with concentration on the Visuddhi chakra, i.e. let him fix his mind unwaveringly upon a point in the spine opposite the larynx. \textsc{Ed [Equinox]})

11. \textit{Seventh practice.} Let the Zelator breathe as deeply and rapidly as possible.

12. \textit{Eighth practice.} Let the Zelator practice restraint of breathing in the following manner.

At any stage of breathing let him suddenly hold the breath, enduring the need to breathe until it passes, returns, and passes again, and so on until consciousness is lost, either rising to Samadhi\index{Samadhi} or similar supernormal condition, or falling into oblivion.

13. \textit{Ninth practice.} Let him practice the usual forms of Pranayama, but let Kumbhakam be used after instead of before expiration. Let him gradually increase the period of this Kumbhakam as in the case of any other.

14. A note concerning the conditions of these experiments.

The conditions favourable are dry and bracing air, a warm climate, absence of wind, absence of noise, insects and all other disturbing influences,\footnote{Note that in the early stages of concentration of the mind, such annoyances become negligible.} a retired situation, simple food eaten in great moderation at the conclusion of the practices of morning and afternoon, and on no account before practising. Bodily health is almost essential, and should be most carefully guarded (See Liber CLXXXV, \textit{Task of a Neophyte}). A diligent and tractable disciple, or the Practicus of the Zelator, should aid him in his work. Such a disciple should be noiseless, patient, vigilant, prompt, cheerful, of gentle manner and reverent to his master, intelligent to anticipate his wants, cleanly and gracious, not given to speech, devoted and unselfish. With all this he should be fierce and terrible to strangers and all hostile influences, determined and vigorous, increasingly vigilant, the guardian of the threshold.

It is not desirable that the Zelator should employ any other creature than a man, save in cases of necessity. Yet for some of these purposes a dog will serve, for others a woman. There are others appointed to serve, but these are not for the Zelator.

15. \textit{Tenth practice.} \textemdash{} Let the Zelator experiment if he will with inhalations of oxygen, nitrous oxide, carbon dioxide, and other gases mixed in small proportion with his air during the practices. These experiments are to be conducted with caution in the presence of a medical man of experience, and they are only useful as facilitating a simulacrum of the results of the proper practices and thereby enheartening the Zelator.

16. \textit{Eleventh practice.} \textemdash{} Let the Zelator at any time during the practices, especially during the periods of Kumbhakam, throw his will utterly towards his Holy Guardian Angel\index{Holy Guardian Angel}, directing his eyes inward and upward, and turning back his tongue as if to swallow it.

(This latter operation is facilitated by severing the fr\ae{}num lingu\ae{}, which, if done, should be done by a competent surgeon. We do not advise this or any similar method of cheating difficulties. This is, however, harmless.)

In this manner the practice is to be raised from the physical to the spiritual plane, even as the words Ruh, Ruach, Pneuma, Spiritus, Geist, Ghost, and indeed words of almost all languages, have been raised from their physical meaning of wind, breath, or movement, to the spiritual plane.

(RV is the old root meaning Yoni and hence Wheel (Fr. roue, Lat. rota, wheel) and the corresponding Semitic root means \enquote{to go}. Similarly spirit is connected with \enquote{spiral}. \textemdash{} \textsc{Ed [Equinox]}.)

17. Let the Zelator attach no credit to any statements that may have been made throughout the course of this instruction, and reflect that even the counsel which we have given as suitable to the average case may be entirely unsuitable to his own.
