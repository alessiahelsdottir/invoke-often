\usepackage{mathalpha}
\usepackage{epigraph}
\usepackage[utf8]{inputenc}
\usepackage[greek.polutoniko, british]{babel} % Sets csquotes language
\usepackage{ulem} % strikeout
\usepackage{graphicx} % For included images in place of inline pentagrams, hexagrams.
\usepackage{array,multirow}
\newcommand\Rotate[1]{\rotatebox[origin=c]{90}{\smaller#1}}
\usepackage{float}
\usepackage{cjhebrew}
\usepackage{amsmath} % For \therefore
\usepackage{amssymb} % For \therefore
\setlength{\voffset}{0.10in}
\usepackage[nf]{coelacanth}
\usepackage{teubner}
\usepackage[LGR, T1]{fontenc}
\let\oldnormalfont\normalfont
\def\normalfont{\oldnormalfont\mdseries}
\usepackage[autostyle=true]{csquotes} % To easily Americanise it.
% \usepackage{ebgaramond} % main font
\usepackage{substitutefont}
\substitutefont{LGR}{\familydefault}{cmr}
\usepackage[final, protrusion]{microtype} % No idea. Supposedly useful.

\usepackage{enumerate}
\usepackage{moreenum}
\setlist{nosep}
\usepackage{ragged2e} % lwenv built from this
\usepackage{varwidth} % For centre verse environment
\usepackage[hang,flushmargin]{footmisc}
\setlength\epigraphwidth{.93\textwidth}
\usepackage[indentfirst=true]{quoting}
\quotingsetup{vskip=0pt}
\usepackage[shortcuts]{extdash}
\usepackage{indentfirst}
\usepackage{xparse}
\usepackage{xpatch}
\newenvironment{tightverse}
  {\vspace{-\itemsep}\begin{verse}}
  {\vspace{-\itemsep}\end{verse}}
\xpatchcmd{\verse}{\itemsep}{\advance\topsep-\parskip\itemsep}{}{}
\usepackage{marvosym}
\setlength{\parskip}{0.0pt}

\addtokomafont{title}{\bfseries\rmfamily}
\renewcommand{\raggedchapter}{\centering\rmfamily\bfseries}
\addtokomafont{section}{\centering\rmfamily\bfseries\fontsize{12}{2}\selectfont}
\addtokomafont{subsection}{\centering\rmfamily\bfseries\fontsize{11}{2}\selectfont}
\addtokomafont{subsubsection}{\centering\rmfamily\bfseries\fontsize{10}{2}\selectfont}
\renewcommand{\raggedsection}{\centering\rmfamily\bfseries}
\addtokomafont{disposition}{\centering\rmfamily\bfseries}

\usepackage{relsize}
\newcommand\chapnum[1]{{\smallskip\bfseries\centering#1\par}}
\usepackage{wasysym}
\newcommand\PParNum[1]{\makebox[-5pt][r]{#1\rule{0pt}{0pt}}  \-\ \-\ \ignorespaces}
% Typesets Hymn to Satan
\newenvironment{Verse}
  {\center\varwidth{\linewidth}\begin{verse}}
  {\end{verse}\endvarwidth\endcenter}

\newcommand\grade[2]{%
$#1^{\circ}=#2^{\square}$
}
% To justify text left and right on the same line
\newcommand\lwenv[2]{%
  \par\smallskip
  \parbox{.5\textwidth}{\begin{RaggedRight}{#1}\end{RaggedRight}}%
  \parbox{.5\textwidth}{\begin{RaggedLeft}{#2}\end{RaggedLeft}}%
  \par}

% For Rahu https://golatex.de/viewtopic.php?t=8193
\usepackage{wasysym}
\usepackage{rotating}
\newcommand*\rahur{%
  \makebox[0pt][l]{\descnode}%
  \reflectbox{\descnode}%
}
\newcommand*\rahu{%
  \makebox[0pt][l]{\ascnode}%
  \reflectbox{\ascnode}%
}
\newcommand\GreekBabalon{Βαβαλον}
\newcommand\Argentium{$A\therefore{}A\therefore$}
\newcommand\GD{$G\therefore{}D\therefore$}
\newcommand\aethyr{\ae{}thyr}
\newcommand\AEthyr{\AE{}thyr}

\DeclareRobustCommand{\editorsnote}[1]{\begin{footnote}{{#1} \textemdash{} \textsc{Editor}.}\end{footnote}}
\newenvironment{Facing}
  {\csname @openrightfalse\endcsname}
  {\csname @openrighttrue\endcsname}
\newenvironment{twopage}{%
  \begingroup\setbox0\vbox\bgroup
}{%
  \egroup
  \ifdim\ht0>\textheight
    \setbox1\vsplit0 to \textheight
    \cleardoublepage\unvbox1\clearpage
    \ifdim\ht0>\textheight
      \PackageWarning{twopage}{Overflow in twopage environment}%
    \fi
    \unvbox0\clearpage
  \else
    \clearpage\unvbox0\clearpage
  \fi\endgroup
}
\usepackage[hidelinks=true,unicode=true,psdextra]{hyperref}
\usepackage{starfont}
\usepackage{tikz}
\usetikzlibrary{shapes.geometric}
\usetikzlibrary{arrows, arrows.meta}
\usetikzlibrary{decorations.markings}
\newcommand\Spirit{\rlap{\Fortune{}}\Terra}
\NewDocumentCommand{\pentagramlabelled}{
  mO{-1}% size
}{
  \begin{tikzpicture}
    \coordinate (0) at (0, #1);
    \node[label=above:{\Spirit [Spirit]}](Spirit) at (0, #1) {};

    \coordinate (1) at ({-cos(54) * #1},{-sin(54) * #1});
    \node[label=below left:{\Earth [Earth]}](Earth) at ({-cos(54) * #1},{-sin(54) * #1}) {};

    \coordinate (2) at ({#1 * cos(18)},{#1 * sin(18)});
    \node[label=right:{\Water [Water]}](Water) at ({#1 * cos(18)},{#1 * sin(18)}) {};

    \coordinate (3) at (-{#1 * cos(18)},{#1 * sin(18)});
    \node[label=left:{\Air [Air]}](Air) at (-{#1 * cos(18)},{#1 * sin(18)}) {};

    \coordinate (5) at ({cos(54) * #1},{sin(54) * -#1});
    \node[label=below right:{\Fire [Fire]}](Fire) at ({cos(54) * #1},{sin(54) * -#1}) {};
    \begin{scope}[thick,decoration={
          markings,
          mark=at position 0.3 with -{\arrow{Latex[length=2mm]}},
          mark=at position 0.9 with -{\arrow{Latex[length=2mm]}}}
      ]

    \ifnum #2=0
    \draw[postaction={decorate}] (Spirit.center) -- (Earth.center);
    \else
    \draw[-] (Spirit.center) -- (Earth.center);
    \fi

    \ifnum #2=1
    \draw[postaction={decorate}] (Earth.center) -- (Water.center);
    \else
    \draw[-] (Earth.center) -- (Water.center);
    \fi

    \ifnum #2=2
    \draw[postaction={decorate}] (Water.center) -- (Air.center);
    \else
    \draw[-] (Water.center) -- (Air.center);
    \fi

    \ifnum #2=3
    \draw[postaction={decorate}] (Air.center) -- (Fire.center);
    \else
    \draw[-] (Air.center) -- (Fire.center);
    \fi

    \ifnum #2=4
    \draw[postaction={decorate}] (Fire.center) -- (Spirit.center);
    \else
    \draw[-] (Fire.center) -- (Spirit.center);
    \fi
    \end{scope}
  \end{tikzpicture}
}
\NewDocumentCommand{\pentagramreversed}{
  mO{-1}% size
}{
  \begin{tikzpicture}
    \coordinate (0) at (0, #1);
    \node(Spirit) at (0, #1) {};

    \coordinate (1) at ({-cos(54) * #1},{-sin(54) * #1});
    \node(Earth) at ({-cos(54) * #1},{-sin(54) * #1}) {};

    \coordinate (2) at ({#1 * cos(18)},{#1 * sin(18)});
    \node(Water) at ({#1 * cos(18)},{#1 * sin(18)}) {};

    \coordinate (3) at (-{#1 * cos(18)},{#1 * sin(18)});
    \node(Air) at (-{#1 * cos(18)},{#1 * sin(18)}) {};

    \coordinate (5) at ({cos(54) * #1},{sin(54) * -#1});
    \node(Fire) at ({cos(54) * #1},{sin(54) * -#1}) {};
    \begin{scope}[thick,decoration={
          markings,
          mark=at position 0.6 with -{\arrow{Latex[length=2mm]}}}
      ]

    \ifnum #2=0
    \draw[postaction={decorate}] (Earth.center) -- (Spirit.center);
    \else
    \draw[-] (Earth.center) -- (Spirit.center);
    \fi

    \ifnum #2=1
    \draw[postaction={decorate}] (Water.center) -- (Earth.center);
    \else
    \draw[-] (Water.center) -- (Earth.center);
    \fi

    \ifnum #2=2
    \draw[postaction={decorate}] (Air.center) -- (Water.center);
    \else
    \draw[-] (Air.center) -- (Water.center);
    \fi

    \ifnum #2=3
    \draw[postaction={decorate}] (Fire.center) -- (Air.center);
    \else
    \draw[-] (Fire.center) -- (Air.center);
    \fi

    \ifnum #2=4
    \draw[postaction={decorate}] (Spirit.center) -- (Fire.center);
    \else
    \draw[-] (Spirit.center) -- (Fire.center);
    \fi
    \end{scope}
  \end{tikzpicture}
}

\NewDocumentCommand{\pentagram}{
  mO{-1}% size
}{
  \begin{tikzpicture}
    \coordinate (0) at (0, #1);
    \node(Spirit) at (0, #1) {};

    \coordinate (1) at ({-cos(54) * #1},{-sin(54) * #1});
    \node(Earth) at ({-cos(54) * #1},{-sin(54) * #1}) {};

    \coordinate (2) at ({#1 * cos(18)},{#1 * sin(18)});
    \node(Water) at ({#1 * cos(18)},{#1 * sin(18)}) {};

    \coordinate (3) at (-{#1 * cos(18)},{#1 * sin(18)});
    \node(Air) at (-{#1 * cos(18)},{#1 * sin(18)}) {};

    \coordinate (5) at ({cos(54) * #1},{sin(54) * -#1});
    \node(Fire) at ({cos(54) * #1},{sin(54) * -#1}) {};
    \begin{scope}[thick,decoration={
          markings,
          mark=at position 0.6 with -{\arrow{Latex[length=2mm]}}}
      ]

    \ifnum #2=0
    \draw[postaction={decorate}] (Spirit.center) -- (Earth.center);
    \else
    \draw[-] (Spirit.center) -- (Earth.center);
    \fi

    \ifnum #2=1
    \draw[postaction={decorate}] (Earth.center) -- (Water.center);
    \else
    \draw[-] (Earth.center) -- (Water.center);
    \fi

    \ifnum #2=2
    \draw[postaction={decorate}] (Water.center) -- (Air.center);
    \else
    \draw[-] (Water.center) -- (Air.center);
    \fi

    \ifnum #2=3
    \draw[postaction={decorate}] (Air.center) -- (Fire.center);
    \else
    \draw[-] (Air.center) -- (Fire.center);
    \fi

    \ifnum #2=4
    \draw[postaction={decorate}] (Fire.center) -- (Spirit.center);
    \else
    \draw[-] (Fire.center) -- (Spirit.center);
    \fi
    \end{scope}
  \end{tikzpicture}
}
\tikzset{
    uptri/.style={
        draw,
        shape border rotate=180,
        regular polygon,
        regular polygon sides=3,
        node distance=2cm,
        minimum height=4em
    }
}
\NewDocumentCommand{\waterhexagram}{
  mO{0}
  }{
  \begin{tikzpicture}[remember picture]
  \begin{scope}[thick,decoration={
        markings,
        mark=at position 0.7 with -{\arrow{Latex[length=2mm]}}}
    ]
  % Upper
  \coordinate (0) at (0, 0);
  \coordinate (1) at (#1, 0);
  \coordinate (2) at (#1/ 2, {(#1 * .86)}); % sqrt(3)/2 ~= 0.86
  \node(A) at (0,0) {};
  \node(B) at (#1, 0) {};
  \node(C) at (#1 / 2, {(#1 * .86)}) {};
  \draw (A.center) -- (B.center);
  \ifnum #2=-1
  \draw[postaction={decorate}] (C.center) -- (B.center) node[label={[xshift=1pt,yshift=2pt]1}]{};
  \else
  \draw (B.center) -- (C.center);
  \fi
  \ifnum #2=1
  \draw[postaction={decorate}] (C.center) -- (A.center) node[label={[xshift=-1pt,yshift=0pt]1}]{};
  \else
  \draw (C.center) -- (A.center);
  \fi

  % Lower
  \coordinate (0) at (0, #1 + .7 * #1);
  \coordinate (1) at (#1, #1 + .7 * #1);
  \coordinate (2) at (#1/ 2, {#1 - (#1 * .86) + .7 * #1}); % sqrt(3)/2 ~= 0.86
  \node(AL) at (0,#1 + .7 * #1) {};
  \node(BL) at (#1, #1 + .7 * #1) {};
  \node(CL) at (#1 / 2, {#1 - (#1 * .86) + .7 * #1}) {};
  \draw (AL.center) -- (BL.center);
  \ifnum #2=-1
  \draw[postaction={decorate}] (CL.center) -- (AL.center) node[label={[xshift=-1pt,yshift=-24pt]2}]{};
  \else
  \draw (AL.center) -- (CL.center);
  \fi
  \ifnum #2=1
  \draw[postaction={decorate}] (CL.center) -- (BL.center) node[label={[xshift=1pt,yshift=-24pt]2}]{};
  \else
  \draw (BL.center) -- (CL.center);
  \fi  \end{scope}
  \end{tikzpicture}
}

\NewDocumentCommand{\airhexagram}{
  mO{0}
  }{
  \begin{tikzpicture}[remember picture]
  \begin{scope}[thick,decoration={
        markings,
        mark=at position 0.7 with -{\arrow{Latex[length=2mm]}}}
    ]
  % Upper
  \coordinate (0) at (0, 0);
  \coordinate (1) at (#1, 0);
  \coordinate (2) at (#1/ 2, {(#1 * .86)}); % sqrt(3)/2 ~= 0.86
  \node(A) at (0,0) {};
  \node(B) at (#1, 0) {};
  \node(C) at (#1 / 2, {(#1 * .86)}) {};
  \draw (A.center) -- (B.center);
  \ifnum #2=-1
  \draw[postaction={decorate}] (C.center) -- (B.center) node[label={[xshift=1pt,yshift=5pt]1}]{};
  \else
  \draw (B.center) -- (C.center);
  \fi
  \ifnum #2=1
  \draw[postaction={decorate}] (C.center) -- (A.center) node[label={[xshift=-1pt,yshift=5pt]1}]{};
  \else
  \draw (C.center) -- (A.center);
  \fi

  % Lower
  \coordinate (0) at (0, #1 - #1);
  \coordinate (1) at (#1, #1 - #1);
  \coordinate (2) at (#1/ 2, {#1 - (#1 * .86) - #1}); % sqrt(3)/2 ~= 0.86
  \node(AL) at (0,#1 - #1) {};
  \node(BL) at (#1, #1 - #1) {};
  \node(CL) at (#1 / 2, {#1 - (#1 * .86) - #1}) {};
  \draw (AL.center) -- (BL.center);
  \ifnum #2=-1
  \draw[postaction={decorate}] (CL.center) -- (AL.center) node[label={[xshift=0pt,yshift=-32pt]2}]{};
  \else
  \draw (AL.center) -- (CL.center);
  \fi
  \ifnum #2=1
  \draw[postaction={decorate}] (CL.center) -- (BL.center) node[label={[xshift=0pt,yshift=-32pt]2}]{};
  \else
  \draw (BL.center) -- (CL.center);
  \fi  \end{scope}
  \end{tikzpicture}
}
\NewDocumentCommand{\earthhexagram}{
  mO{0}
  }{
  \begin{tikzpicture}[remember picture]
  \begin{scope}[thick,decoration={
        markings,
        mark=at position 0.7 with -{\arrow{Latex[length=2mm]}}}
    ]
  % Upper
  \coordinate (0) at (0, 0);
  \coordinate (1) at (#1, 0);
  \coordinate (2) at (#1/ 2, {(#1 * .86)}); % sqrt(3)/2 ~= 0.86
  \node(A) at (0,0) {};
  \node(B) at (#1, 0) {};
  \node(C) at (#1 / 2, {(#1 * .86)}) {};
  \draw (A.center) -- (B.center);
  \ifnum #2=-1
  \draw[postaction={decorate}] (C.center) -- (B.center) node[label={[xshift=-7pt,yshift=10pt]1}]{};
  \else
  \draw (B.center) -- (C.center);
  \fi
  \ifnum #2=1
  \draw[postaction={decorate}] (C.center) -- (A.center) node[label={[xshift=5pt,yshift=10pt]1}]{};
  \else
  \draw (C.center) -- (A.center);
  \fi

  % Lower
  \coordinate (0) at (0, #1 - 0.5);
  \coordinate (1) at (#1, #1 - 0.5);
  \coordinate (2) at (#1/ 2, {#1 - (#1 * .86) - 0.5}); % sqrt(3)/2 ~= 0.86
  \node(AL) at (0,#1 - 0.5) {};
  \node(BL) at (#1, #1 - 0.5) {};
  \node(CL) at (#1 / 2, {#1 - (#1 * .86) - 0.5}) {};
  \draw (AL.center) -- (BL.center);
  \ifnum #2=-1
  \draw[postaction={decorate}] (CL.center) -- (AL.center) node[label={[xshift=5pt,yshift=-32pt]2}]{};
  \else
  \draw (AL.center) -- (CL.center);
  \fi
  \ifnum #2=1
  \draw[postaction={decorate}] (CL.center) -- (BL.center) node[label={[xshift=-4pt,yshift=-32pt]2}]{};
  \else
  \draw (BL.center) -- (CL.center);
  \fi  \end{scope}
  \end{tikzpicture}
}

\NewDocumentCommand{\firehexagram}{
  mO{0}
  }{
  \begin{tikzpicture}[remember picture]
  \begin{scope}[thick,decoration={
        markings,
        mark=at position 0.7 with -{\arrow{Latex[length=2mm]}}}
    ]
  % Upper
  \coordinate (0) at (0, 0 + .5);
  \coordinate (1) at (#1, 0 + .5);
  \coordinate (2) at (#1/ 2, {.5 + (#1 * .86)}); % sqrt(3)/2 ~= 0.86
  \node(A) at (0,0 + .5) {};
  \node(B) at (#1, 0 + .5) {};
  \node(C) at (#1 / 2, {.5 + (#1 * .86)}) {};
  \draw (A.center) -- (B.center);
  \ifnum #2=-1
  \draw[postaction={decorate}] (C.center) -- (B.center) node[label={[xshift=4pt,yshift=1pt]1}]{};
  \else
  \draw (B.center) -- (C.center);
  \fi
  \ifnum #2=1
  \draw[postaction={decorate}] (C.center) -- (A.center) node[label={[xshift=-4pt,yshift=1pt]1}]{};
  \else
  \draw (C.center) -- (A.center);
  \fi

  % Lower

  \coordinate (0) at (0, 0);
  \coordinate (1) at (#1, 0);
  \coordinate (2) at (#1/ 2, #1 * .86); % sqrt(3)/2 ~= 0.86
  \node(AL) at (0,0) {};
  \node(BL) at (#1, 0) {};
  \node(CL) at (#1 / 2, #1 * .86) {};
  \draw (AL.center) -- (BL.center);
  \ifnum #2=-1
  \draw[postaction={decorate}] (CL.center) -- (BL.center) node[label={[xshift=4pt,yshift=1pt]2}]{};
  \else
  \draw (BL.center) -- (CL.center);
  \fi
  \ifnum #2=1
  \draw[postaction={decorate}] (CL.center) -- (AL.center) node[label={[xshift=-4pt,yshift=1pt]2}]{};
  \else
  \draw (CL.center) -- (AL.center);
  \fi
  \end{scope}
  \end{tikzpicture}
}
