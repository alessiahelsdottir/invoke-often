\addchap{Liber Jugorum}\index{Grade Studies \& Work!Probationer (\grade{0}{0})}\index{Grade Studies \& Work!Zelator (\grade{2}{9})}
\chapnum{III\footnote{The title may be read as \enquote{Book of the Yoke}. Refers to the threefold method given, and to the Triangle as a binding force. From the Equinox Vol. 1, No. 4 \& Magick in Theory and Practice.}}

\textbf{An instruction for the control of speech, action, and thought.}
\addsec*{O}

0. Behold the Yoke upon the neck of the Oxen! Is it not thereby that the Field shall be ploughed? The Yoke is heavy, but joineth together them that are separate \textemdash{} Glory to Nuit and to Hadit, and to Him that hath given us the Symbol of the Rosy Cross!

Glory unto the Lord of the Word Abrahadabra, and Glory unto Him that hath given us the Symbol of the Ankh, and of the Cross within the Circle!

1. Three are the Beasts wherewith thou must plough the Field; the Unicorn, the Horse, and the Ox. And these shalt thou yoke in a triple yoke that is governed by One Whip.

2. Now these Beasts run wildly upon the earths and are not easily obedient to the Man.

3. Nothing shall be said here of Cerberus, the great Beast of Hell that is every one of these and all of these, even as Athanasius hath foreshadowed. For this matter\footnote{(I.e. the matter of Cereberus)} is not of Tiphereth without, but Tiphereth within.

\addsec*{I}\index{Control of!Speech}\index{Grade Studies \& Work!Practicus (\grade{3}{8})}
0. The Unicorn is speech. Man, rule thy Speech! How else shalt thou master the Son, and answer the Magician at the right hand gateway of the Crown?

1. Here are practices. Each may last for a week or more.
\begin{enumerate}[label=\greek*.]
\item Avoid using some common word, such as \enquote{and} or \enquote{the} or \enquote{but}; use a paraphrase.

\item Avoid using some letter of the alphabet, such as \enquote{t}, or \enquote{s}, or \enquote{m}; use a paraphrase.

\item Avoid using the pronouns and adjectives of the first person; use a paraphrase.
\end{enumerate}
Of thine own ingenium devise others.

2. On each occasion that thou art betrayed into saying that thou art sworn to avoid, cut thyself sharply upon the writes or forearm with a razor\editorsnote{Less permanent, damaging alternatives to the razor exist like making a note of the failure, and/or using a rubber band to punish. The goal here is to train yourself to be mindful of speech, action, and thought, not to butcher yourself.}\editorsnote{If one does partake in these practices, remember to sterilise the blade, to practice first aid, and keep in mind how it will heal and scar.}; even as thou shouldst beat a disobedient dog. Feareth not the Unicorn the claws and teeth of the Lion?

3. Thine arm then serveth thee both for a warning and for a record. Thou shalt write down thy daily progress in these practices, until thou art perfectly vigilant at all times over the least word that slippeth from thy tongue.

Thus bind thyself, and thou shalt be for ever free.

\addsec*{II}\index{Control of!Action}\index{Grade Studies \& Work!Philosophus (\grade{4}{7})}
0.The Horse is Action. Man, rule thine Action. How else shalt thou master the Father, and answer the Fool at the Left Hand Gateway of the Crown?

1. Here are practices. Each may last for a week, or more.

\begin{enumerate}[label=\greek*.]
\item Avoiding lifting the left arm above the waist.

\item Avoid crossing the legs.
\end{enumerate}

Of thine own ingenium devise others.

2. On each occasion that thou art betrayed into doing that thou art sworn to avoid, cut thyself sharply upon the wrist or forearm with a razor; even as thou shouldst beat a disobedient dog. Feareth not the Horse the teeth of the Camel?

3. Thine arm then serveth thee both for a warning and for a record. Thou shalt write down thy daily progress in these practices, until thou art perfectly vigilant at all times over the least action that slippeth from the least of thy fingers.

Thus bind thyself, and thou shalt be for ever free.

\addsec*{III}\index{Control of!Thoughts and Personality}\index{Grade Studies \& Work!Dominus Liminis}
0. The Ox is Thought. Man, rule thy Thought! How else shalt thou master the Holy Spirit, and answer the High Priestess in the Middle Gateway of the Crown?

1. Here are practices. Each may last for a week or more.

\begin{enumerate}[label=\greek*.]
\item Avoid thinking of a definite subject and all things connected with it, and let that subject be one which commonly occupies much of thy thought, being frequently stimulated by sense-perceptions or the conversation of others.

\item By some device, such as the changing of thy ring from one finger to another, create in thyself two personalities, the thoughts of one being within entirely different limits from that of the other, the common ground being the necessities of life.\footnote{For instance, let A be a man of strong passions, skilled in the Holy Qabalah, a vegetarian, and a keen \enquote{reactionary} politician; let B be a bloodless and ascetic thinker, occupied with business and family cares, an eater of meat, and a keen progressive politician. Let no thought proper to \enquote{A} arise when the ring is on the \enquote{B} finger; and \textit{vice versa}.}
\end{enumerate}

Of thine own Ingenium devise others.

2. On each occasion that thou art betrayed into thinking that thou art sworn to avoid, cut thyself sharply upon the wrist or forearm with a razor; even as thou shouldst beat a disobedient dog. Feareth not the Ox the Goad of the Ploughman?

3. Thine arm then serveth thee both for a warning and for a record. Thou shalt write down thy daily progress in these practices, until thou art perfectly vigilant at all times over the least thought that ariseth in thy brain.

Thus bind thyself, and thou shalt be for ever free.
