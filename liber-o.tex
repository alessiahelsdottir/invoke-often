\addchap{Liber O vel Manus et Sagitt\ae{} [Abridged]}
\chapnum{VI\footnote{Equinox Vol. 1, No. 2; Magick in Theory and Practice.}}
\textbf{Instructions given for elementary study of the Qabalah, Assumption of God forms, Vibration of Divine Names, the Rituals of Pentagram and Hexagram, and their uses in protection and invocation, a method of attaining astral visions so-called, and an instruction in the practice called Rising on the Planes.}

\addsec*{I. [Introduction]}

1. This book is very easy to misunderstand; readers are asked to use the most minute critical care in the study of it, even as we have done in its preparation.

2. In this book it is spoken of the Sephiroth and the Paths; of Spirits and Conjurations; of Gods, Spheres, Planes, and many other things which may or may not exist.

It is immaterial whether these exist or not. By doing certain things certain results will follow; students are most earnestly warned against attributing objective reality or philosophic validity to any of them.

3. The advantages to be gained from them are chiefly these:

\begin{enumerate}[label=(\textit{\alph*})]
\item A widening of the horizon of the mind.
\item An improvement of the control of the mind.
\end{enumerate}

4. The student, if he attains any success in the following practices, will find himself confronted by things (ideas or beings) too glorious or too dreadful to be described. It is essential that he remain the master of all that he beholds, hears or conceives; otherwise he will be the slave of illusion, and the prey of madness.

Before entering upon any of these practices, the student should be in good health, and have attained a fair mastery of Asana, Pranayama and Dharana.

5. There is little danger that any student, however idle or stupid, will fail to get some result; but there is great danger that he will be led astray, obsessed and overwhelmed by his results, even though it be by those which it is necessary that he should attain. Too often, moreover, he mistaketh the first resting-place for the goal, and taketh off his armour as if he were a victor ere the fight is well begun.

It is desirable that the student should never attach to any result the importance which it at first seems to possess.

6. First, then, let us consider the Book 777 and its use; the preparation of the Place; the use of the Magic Ceremonies; and finally the methods which follow in Chapter V. \enquote{Viator in Regnis Arboris}, and in Chapter VI. \enquote{Sagitta trans Lunam}.

(In another book will it be treated of the Expansion and Contraction of Consciousness; progress by slaying the Chakkr\^{a}ms; progress by slaying the Pairs of Opposites; the methods of Sabhapaty Swami, \&c. \&c.)


\addsec*{II. [The Use of 777 \& Qabalistic correspondence essentials]}


1. The student must \textsc{first} obtain a thorough knowledge of Book 777, especially of columns i., ii., iii., v., vi., vii., ix., xi., xii., xiv., xv., xvi., xvii., xviii., xix., xxxiv., xxxv., xxxviii., xxxix., xl., xli., xlii., xlv., liv., lv., lix., lx., lxi., lxiii., lxx., lxxv., lxxvii., lxviii., lxxix., lxxx., lxxxi., lxxxiii., xcvii., xcviii., xcix., c., ci., cxvii., cxviii., cxxxvii., cxxxviii., cxxxix., clxxv., clxxvi., clxxvii., clxxxii.

When these are committed to memory, he will begin to understand the nature of these correspondences. (\textit{See} Illustrations: \enquote{The Temple of Solomon the King} in this number\footnote{Equinox Vol. 1, No. 2.}. Cross-references are given.)

2. If we take an example, the use of the table will become clear. Let us suppose that you wish to obtain knowledge of some obscure science. In column xlv., line 12, you will find \enquote{Knowledge of Sciences}.

By now looking up line 12 in the other columns, you will find that the Planet corresponding is Mercury, its number eight, its lineal figures the octagon and octagram. The God who rules that planet Thoth, or in Hebrew symbolism Tetragrammaton Adonai and Elohim Tzabaoth, its Archangel Raphael, its Choir of Angels Beni Elohim, its Intelligence Tiriel, its Spirit Taphtatharath, its colours Orange (for Mercury is the Sphere of the Sephira Hod, 8), Yellow, Purple, Grey, and Indigo rayed with Violet; its Magical Weapon the Wand or Caduceus, its Perfumes Mastic and others, its sacred plants Vervain and others, its jewel the Opal or Agate; its sacred animal the Snake, \&c., \&c.

3. You would then prepare your Place of Working accordingly. In an orange circle you would draw an eight-pointed star of yellow, at whose points you would place eight lamps. The Sigil of the Spirit (which is to be found in Cornelius Agrippa and other books) you would draw in the four colours with such other devices as your experience may suggest.

4. And so on. We cannot here enter at length into all the necessary preparations; and the student will find them fully set forth in the proper books, of which the \enquote{Goetia} is perhaps the best example.

These rituals need not be slavishly imitated; on the contrary the student should do nothing the object of which he does not understand; also, if he have any capacity whatever, he will find his own crude rituals more effective than the highly polished ones of other people.


The general purpose of all this preparation is as follows:


5. Since the student is a man surrounded by material objects, if it be his wish to master one particular idea, he must make every material object about him directly suggest that idea. Thus in the ritual quoted, if his glance fall upon the lights, their number suggests Mercury; he smells the perfumes, and again Mercury is brought to his mind. In other words, the whole magical apparatus and ritual is a complex system of mnemonics.

(The importance of these lies principally in the fact that particular sets of images that the student may meet in his wanderings correspond to particular lineal figures, divine names, \&c. and are controlled by them. As to the possibility of producing results external to the mind of the seer (\textit{objective}, in the ordinary common-sense acceptation of the term) we are here silent.)

6. There are three important practices connected with all forms of ceremonial (and the two Methods which later we shall describe). These are:

\begin{enumerate}[label=(\arabic*)]
\item Assumption of God-forms.
\item Vibration of Divine Names.
\item Rituals of \enquote{Banishing} and \enquote{Invoking}.
\end{enumerate}

These, at least, should be completely mastered before the dangerous Methods of Chapters V. and VI. are attempted.


\addsec*{III. [Assumption of God-forms \& Vibration of Names]}


1. The Magical Images of the Gods of Egypt should be made thoroughly familiar. This can be done by studying them in any public museum, or in such books as may be accessible to the student. They should then be carefully painted by him, both from the model and from memory.

2. The student, seated in the \enquote{God} position or in the characteristic attitude of the God desired, should then imagine His image as coinciding with his own body, or as enveloping it. This must be practiced until mastery of the image is attained, and an identity with it and with the God experienced.

It is a matter for very great regret that no simple and certain test of success in this practice exists.

3. The Vibration of God-names. As a further means of identifying the human consciousness with that pure portion of it which man calls by the name of some God, let him act thus:

4. (\textit{a}) \-\ Stand with arms outstretched.\footnote{This injunction does not apply to gods like Phthah or Harpocrates whose natures do not accord with this gesture.} \begin{enumerate}[start=2, label=(\textit{\alph*}), leftmargin=3.57\parindent]
\item Breathe in deeply through the nostrils, imagining the name of the God desired entering with the breath.
\item Let that name descend slowly from the lungs to the heart, the solar plexus, the navel, the generative organs, and so to the feet.
\item The moment that it appears to touch the feet, quickly advance the left foot about 12 inches, throw forward the body, and let the hands (drawn back to the side of the eyes) shoot out, so that you are standing in the typical position of the God Horus\footnote{\textit{See} Illustration in [Equinox] Vol., I. No. 1, \enquote{Blind Force}.} and at the same time imagine the Name as rushing up and through the body, while you breathe it out through the nostrils with the air which has been till then retained in the lungs. All this must be done with all the force of which you are capable.
\item Then withdraw the left foot\footnote{Or the thumb, the fingers being closed. The thumb symbolises spirit, the forefinger the element of water.}, and place the right forefinger upon the lips, so that you are in the characteristic position of the God Harpocrates\footnote{\textit{See} Illustration in [Equinox] Vol. I., No. 1, \enquote{The Silent Watcher}.}
\end{enumerate}

5. It is a sign that the student is performing this correctly when a single \enquote{Vibration} entirely exhausts his physical strength. It should cause him to grow hot all over, or to perspire violently, and it should so weaken him that he will find it difficult to remain standing.

6. It is a sign of success, though only by the student himself is it perceived, when he hears the name of the God vehemently roared forth, as if by the concourse of ten thousand thunders; and it should appear to him as if that Great Voice proceeded from the Universe, and not from himself.

In both the above practices all consciousness of anything but the God-form and name should be absolutely blotted out; and the longer it takes for normal perception to return, the better.


\addsec*{IV. [Rituals]}

1. The Rituals of the Pentagram and Hexagram\editorsnote{Please see the appendices for images of the pentagrams and hexagrams.} must be committed to memory; these are as follows:
\subsection*{The Greater Rituals of the Pentagram and Hexagram}
\textit{\nopagebreak Omitted. They are not quite in the form of instruction in Liber O and thus are not in line with this intention of this text.  The elemental pentagrams themselves are in the Appendix.}

\subsection*{The Lesser Ritual of the Pentagram}
\begin{enumerate}[label=(\Roman*)]
\item Touching the forehead say Ateh (Unto Thee).
\item Touching the breast say Malkuth (The Kingdom).
\item Touching the right shoulder, say ve-Geburah (and the Power).
\item Touching the left shoulder, say ve-Gedulah (and the Glory).
\item Clasping the hands upon the breast, say le-Olahm, Amen (To the Ages, Amen).
\item Turning to the East make a pentagram (that of Earth)\editorsnote{Usually considered to appropriate for invoking and banishing by changing the direction of the Earth pentagram. See e.g. Israel Regardie's \enquote{Golden Dawn}.} with the proper weapon (usually the Wand). Say (\textit{i.e.} vibrate) \linebreak[2] I H V H.
\item Turning to the South, the same, but say \linebreak[2] A D N I.
\item Turning to the West, the same, but say \linebreak[2] A H I H.
\item Turning to the North, the same, but say \linebreak[2] A G L A.
\item[] Pronounce: Ye-ho-wau, Ad\'{o}nai, Eheieh, Agla.
\item Extending the arms in the form of a Cross say:
\item Before me Raphael;
\item Behind me Gabriel;
\item On my right hand Michael.
\item On my left hand Auriel;
\item For about me flames the Pentagram,
\item And in the Column stands the six-rayed Star.
\item[] Repeat (i) to (v), the \enquote{Qabalistic Cross}.
\end{enumerate}

\subsection*{The Lesser Ritual of the Hexagram.}

This ritual is to be performed after the \enquote{Lesser Ritual of the Pentagram}.
\begin{enumerate}[label=(\textit{\Roman*})]
\item Stand upright, feet together, left arm at side, right across body, holding the wand or other weapon upright in the median line. Then face East and say:
\item I.N.R.I. \\
Yod. Nun. Resh. Yod. \\
Virgo, Isis, Mighty Mother. \\
Scorpio, Apophis, Destroyer. \\
Sol, Osiris, Slain and Risen. \\
Isis, Apophis, Osiris, IAO.
\item Extend the arms in the form of a cross [\Cross \textendash{} \enquote{The Cross}], and say: \enquote{The Sign of Osiris Slain}.\editorsnote{Example images are in the Equinox Vol. 1, No. 2, and readily available online.}
\item Raise the right arm to point upwards, keeping the elbow square, and lower the left arm to point downwards, keeping the elbow square, while turning the head over the left shoulder looking down so that the eyes follow the left forearm [$L$ \textendash{} \enquote{The Swastika}], and say, \enquote{The sign of the Mourning of Isis}.
\item Raise the arms at an angle of sixty degrees to each other above the head, which is thrown back [$V$ \textendash{} \enquote{The Trident}], and say, \enquote{The Sign of Apophis and Typhon}.
\item Cross the arms on the breast, and bow the head [$X$ \textendash{} \enquote{The Pentagram}] and say, \enquote{The Sign of Osiris Risen}.
\item Extend the arms again as in (iii) and cross them again as in (vi) saying: \enquote{L.V.X., Lux, the Light of the Cross}.
\item With the magical weapon trace the Hexagram of Fire\footnote{This hexagram consists of two equilateral triangles, both apices pointed upwards. Begin at the top of the upper triangle and trace it in a dextro-rotary direction. The top of the lower triangle should coincide with the central point of the upper triangle.} in the East saying: \enquote{ARARITA}.\footnote{\cjRL{'r'ryt'}\footnotemark}\footnotetext{Which word consists of the initials of a sentence which means \enquote{One is His Beginning: One is His Individuality: His Permutation is One}.}
\item Trace the Hexagram of Earth\footnote{This Hexagram has the apex of the lower triangle pointing downwards, and it should be capable of inscription in a circle.} in the South saying: \enquote{ARARITA.}
\item Trace the Hexagram of Air\footnote{This hexagram is like that of Earth; but the bases of the triangles coincide, forming a diamond.} in the West saying: \\ \enquote{ARARITA.}
\item Trace the Hexagram of Water\footnote{This hexagram has the lower triangle placed above the upper, so that their apices coincide.} in the North saying: \enquote{ARARITA.}
\item Repeat (i-vii)
\item[] The Banishing Ritual is identical, save that the direction of the Hexagrams must be reversed.
\end{enumerate}

\addsec*{IV. [Continued]}

2. These rituals should be practiced until the figures drawn appear in flame, in flame so near to physical flame that it would perhaps be visible to the eyes of a bystander, were one present. It is alleged that some persons have attained the power of actually kindling fire by these means. Whether this be so or not, the power is not one to be aimed at.

3. Success in \enquote{banishing} is known by a \enquote{feeling of cleanliness} in the atmosphere; success in \enquote{invoking} by a \enquote{feeling of holiness}. It is unfortunate that these terms are so vague.

But at least make sure of this: that any imaginary figure or being shall instantly obey the will of the student, when he uses the appropriate figure. In obstinate cases, the form of the appropriate God may be assumed.

4. The banishing rituals should be used at the commencement of any ceremony whatever. Next, the student should use a general invocation, such as the \enquote{Preliminary Invocation} in the \enquote{Goetia} as well as a special invocation to suit the nature of his working.

5. Success in these verbal invocations is so subtle a matter, and its grades so delicately shaded, that it must be left to the good sense of the student to decide whether or not he should be satisfied with his result.

\addsec*{V. [The Body of Light \& Astral Projection]}


1. Let the student be at rest in one of his prescribed positions, having bathed and robed with the proper decorum. Let the place of working be free from all disturbance, and let the preliminary purifications, banishings and invocations be duly accomplished, and, lastly, let the incense be kindled.

2. Let him imagine his own figure (preferably robed in the proper magical garments and armed with the proper magical weapons) as enveloping his physical body, or standing near to and in front of him.

3. Let him then transfer the seat of his consciousness to that imagined figure; so that it may seem to him that he is seeing with its eyes, and hearing with its ears.

This will usually be the great difficulty of the operation.

4. Let him then cause that imagined figure to rise in the air to a great height above the earth.

5. Let him then stop and look about him. (It is sometimes difficult to open the eyes.)

6. Probably he will see figures approaching him, or become conscious of a landscape.

Let him speak to such figures, and insist upon being answered, using the proper pentagrams and signs, as previously taught.

7. Let him travel about at will, either with or without guidance from such figure or figures.

8. Let him further employ such special invocations as will cause to appear the particular places he may wish to visit.

9. Let him beware of the thousand subtle attacks and deceptions that he will experience, carefully testing the truth of all with whom he speaks.

Thus a hostile being may appear clothed with glory; the appropriate pentagram will in such a case cause him to shrivel or decay.

10. Practice will make the student infinitely wary in these matters.

11. It is usually quite easy to return to the body, but should any difficulty arise, practice (again) will make the imagination fertile. For example, one may create in thought a chariot of fire with white horses, and command the charioteer to drive earthwards.

It might be dangerous to go too far, or to stay too long; for fatigue must be avoided.

The danger spoken of is that of fainting, or of obsession, or of loss of memory or other mental faculty.

12. Finally, let the student cause his imagined body in which he supposes himself to have been travelling to coincide with the physical, tightening his muscles, drawing in his breath, and putting his forefinger to his lips. Then let him \enquote{awake} by a well-defined act of will, and soberly and accurately record his experiences.

It may be added that this apparently complicated experiment is perfectly easy to perform. It is best to learn by \enquote{travelling} with a person already experienced in the matter. Two or three experiments will suffice to render the student confident and even expert. \textit{See also} \enquote{The Seer}, pp. 295-333.


\addsec*{VI. [Rising on the Planes]}


1. The previous experiment has little value, and leads to few results of importance. But it is susceptible of a development which merges into a form of Dharana \textemdash{} concentration \textemdash{} and as such may lead to the very highest ends. The principal use of the practice in the last chapter is to familiarise the student with every kind of obstacle and every kind of delusion, so that he may be perfect master of every idea that may arise in his brain, to dismiss it, to transmute it, to cause it instantly to obey his will.

2. Let him then begin exactly as before, but with the most intense solemnity and determination.

3. Let him be very careful to cause his imaginary body to rise in a line exactly perpendicular to the earth's tangent at the point where his physical body is situated (or to put it more simply, straight upwards).

4. Instead of stopping, let him continue to rise until fatigue almost overcomes him. If he should find that he has stopped without willing to do so, and that figures appear, let him at all costs rise above them.

Yea, though his very life tremble on his lips, let him force his way upward and onward!

5. Let him continue in this so long as the breath of life is in him. Whatever threatens, whatever allures, though it were Typhon and all his hosts loosed from the pit and leagued against him, though it were from the very Throne of God Himself that a Voice issues bidding him stay and be content, let him struggle on, ever on.

6. At last there must come a moment when his whole being is swallowed up in fatigue, overwhelmed by its own inertia.\footnote{This in case of failure. The results of success are so many and wonderful that no effort is here made to describe them. They are classified, tentatively, in the \enquote{Herb Dangerous}, Part II, Equinox Vol. 1, No. 2.} Let him sink (when no longer can he strive, though his tongue by bitten through with the effort and the blood gush from his nostrils) into the blackness of unconsciousness; and then, on coming to himself, let him write down soberly and accurately a record of all that hath occurred, yea a record of all that hath occurred.

{
\centering

\textsc{EXPLICIT}
\par
}
