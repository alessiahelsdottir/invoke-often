\addchap{The Mass of the Phoenix}
\chapnum{XLIV\footnote{The Book of Lies; Magick in Theory and Practice.}}

\epigraph{This mass should be performed daily at sunset by every magician.}{Aleister Crowley, Magick in Theory and Practice, Chapter XX.}

\textbf{A Ritual of the Law.}

\textit{The Magician, his breast bare, stands before an altar on which are his Burin, Bell, Thurible, and two of the Cakes of Light. In the Sign of the Enterer he reaches West across the Altar, and cries:}

\begin{quoting}[indentfirst=false]
Hail Ra, that goest in Thy bark \\
Into the Caverns of the Dark!
\end{quoting}

\textit{He gives the sign of Silence, and takes the Bell, and Fire, in his hands.}

\begin{quoting}[indentfirst=false]
East of the Altar see me stand \\
With Light and Musick in my hand!
\end{quoting}

\textit{He strikes Eleven times upon the Bell 333 - 55555 - 333 and places the Fire in the Thurible.}

\begin{quoting}[indentfirst=false]
I strike the Bell: I light the Flame; \\
I utter the mysterious Name. \\
\textsc{Abrahadabra}
\end{quoting}

\textit{He strikes eleven times upon the Bell.}

\begin{quoting}[indentfirst=false]
Now I begin to pray: Thou Child, \\
Holy Thy name and undefiled! \\
Thy reign is come: Thy will is done. \\
Here is the Bread; here is the Blood. \\
Bring me through midnight to the Sun! \\
Save me from Evil and from Good! \\
That Thy one crown of all the Ten \\
Even now and here be mine. \textsc{Amen}.
\end{quoting}

\textit{He puts the first Cake on the Fire of the Thurible.}

\begin{quoting}[indentfirst=false]
I burn the Incense-cake, proclaim \\
These adorations of Thy name. 
\end{quoting}

\textit{He makes them as in Liber Legis, and strikes again Eleven times upon the Bell. With the Burin he then makes upon his breast the proper sign.}\editorsnote{Make the Mark of the Beast (\textit{see} appendix) or a cross in a circle (e.g. $\bigoplus$ or $\bigotimes$).}\editorsnote{Lon Milo DuQuette suggests a machinist's scribe in his book the Magick of Aleister Crowley, stating that it is easy to sterilise and use to only lightly scratch. The mark itself may be a non-bleeding scratch, and in the end, you need but the tiniest drop of blood to fulfil the Eucharist.}

\begin{quoting}[indentfirst=false]
Behold this bleeding breast of mine \\
Gashed with the sacramental sign! 
\end{quoting}

\textit{He puts the second Cake to the wound.}

\begin{quoting}[indentfirst=false]
I stanch the Blood; the wafer soaks \\
It up, and the high priest invokes!
\end{quoting}

\textit{He eats the second Cake.}

\begin{quoting}[indentfirst=false]
This Bread I eat. This Oath I swear \\
As I enflame myself with prayer: \\
\enquote{There is no grace: there is no guilt: \\
This is the Law: \textsc{Do What Thou Wilt!}}
\end{quoting}

\textit{He strikes Eleven times upon the Bell, and cries} \textsc{Abrahadabra}

\begin{quoting}[indentfirst=false]
I entered in with woe; with mirth \\
I now go forth, and with thanksgiving, \\
To do my pleasure on the earth \\
Among the legions of the living.
\end{quoting}

\textit{He goeth forth.}
