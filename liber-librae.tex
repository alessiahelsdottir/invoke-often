\addchap{Liber Libr\ae{}}\index{Grade Studies \& Work!Probationer (\grade{0}{0})}\index{Grade Studies \& Work!Adeptus Minor (\grade{5}{6})}

\chapnum{XXX\footnote{30 is the letter Lamed, which is Justice\footnotemark in the Tarot, referred to Libra. From the Equinox, Vol. 1, No. 1.}}\footnotetext{Liber Libr\ae{} may be read as \enquote{The Book of Scales} or \enquote{The Book of Balances}. --- \textsc{Editor}.}
0. Learn first---Oh thou who aspirest unto our ancient Order!---that Equilibrium\index{Equilibrium} is the basis of the Work. If thou thyself hast not a sure foundation, whereon wilt thou stand to direct the forces of Nature?

1. Know then, that as man is born into this world amidst the Darkness of Matter, and the strife of contending forces; so must his first endeavour be to seek the Light through their reconciliation.

2. Thou then, who hast trials and troubles, rejoice because of them, for in them is Strength, and by their means is a pathway opened unto that Light.

3. How should it be otherwise, O man, whose life is but a day in Eternity, a drop in the Ocean of time; how, were thy trials not many, couldst thou purge thy soul from the dross of earth?

Is it but now that the Higher Life is beset with dangers and difficulties; hath it not ever been so with the Sages and Hierophants of the past? They have been persecuted and reviled, they have been tormented of men; yet through this also has their Glory increased.

4. Rejoice therefore, O Initiate, for the greater thy trial the greater thy Triumph. When men shall revile thee, and speak against thee falsely, hath not the Master said, \enquote{Blessed art thou!}?

5. Yet, oh aspirant, let thy victories bring thee not Vanity, for with increase of Knowledge should come increase of Wisdom. He who knoweth little, thinketh he knoweth much; but he who knoweth much hath learned his own ignorance. Seest thou a man wise in his own conceit? There is more hope of a fool, than of him.

6. Be not hasty to condemn others; how knowest thou that in their place, thou couldest have resisted the temptation? And even were it so, why shouldst thou despise one who is weaker than thyself?

7. Thou therefore who desirest Magical Gifts, be sure that thy soul is firm and steadfast; for it is by flattering thy weaknesses that the Weak Ones will gain power over thee. Humble thyself before thy Self, yet fear neither man not spirit. Fear is failure, and the forerunner of failure: and courage is the beginning of virtue.

8. Therefore fear not the Spirits, but be firm and courteous with them; for thou hast no right to despise or revile them; and this too may lead thee astray. Command and banish them, curse them by the Great Names if need be; but neither mock nor revile them, for so assuredly wilt thou be lead into error.

9. A man is what he maketh himself within the limits fixed by his inherited destiny; he is a part of mankind; his actions affect not only what he calleth himself, but also the whole universe.

10. Worship and neglect not, the physical body which is thy temporary connection with the outer and material world. Therefore let thy mental Equilibrium be above disturbance by material events; strengthen and control the animal passions, discipline the emotions and the reason, nourish the Higher Aspirations.\index{Physical work}

11. Do good unto others for its own sake, not for reward, not for gratitude from them, not for sympathy. If thou art generous, thou wilt not long for thine ears to be tickled by expressions of gratitude.

12. Remember that unbalanced force is evil; that unbalanced severity is but cruelty and oppression; but that also unbalanced mercy is but weakness which would allow and abet Evil. Act passionately; think rationally; be Thyself.

13. True ritual is as much action as word; it is Will.

14. Remember that this earth is but an atom in the universe, and that thou thyself art but an atom thereon, and that even couldst thou become the God of this earth whereon thou crawlest and grovellest, that thou wouldest, even then, be but an atom, and one amongst many.

15. Nevertheless have the greatest self-respect, and to that end sin not against thyself. The sin which is unpardonable is knowingly and wilfully to reject truth, to fear knowledge lest that knowledge pander not to thy prejudices.

16. To obtain Magical Power, learn to control thought; admit only those ideas that are in harmony with the end desired, and not every stray and contradictory Idea that presents itself.

17. Fixed thought is a means to an end. Therefore pay attention to the power of silent thought and meditation. The material act is but the outward expression of thy thought, and therefore hath it been said that \enquote{the thought of foolishness is sin.} Thought is the commencement of action, and if a chance thought can produce much effect, what cannot fixed thought do?

18. Therefore, as hath already been said, Establish thyself firmly in the equilibrium of forces, in the centre of the Cross of the Elements, that Cross from whose centre the Creative Word issued in the birth of the Dawning Universe.

19. Be thou therefore prompt and active as the Sylphs, but avoid frivolity and caprice; be energetic and strong like the Salamanders, but avoid irritability and ferocity; be flexible and attentive to images like the Undines, but avoid idleness and changeability; be laborious and patient like the Gnomes, but avoid grossness and avarice.

20. So shalt thou gradually develop the powers of thy soul, and fit thyself to command the Spirits of the elements. For wert thou to summon the Gnomes to pander to thine avarice, thou wouldst no longer command them, but they would command thee. Wouldst thou abuse the pure beings of the woods and mountains to fill thy coffers and satisfy thy hunger of Gold? Wouldst thou debase the Spirits of Living Fire to serve thy wrath and hatred? Wouldst thou violate the purity of the Souls of the Waters to pander to thy lust of debauchery? Wouldst thou force the Spirits of the Evening Breeze to minister to thy folly and caprice? Know that with such desires thou canst but attract the Weak, not the Strong, and in that case the Weak will have power over thee.

21. In the true religion there is no sect, therefore take heed that thou blaspheme not the name by which another knoweth his God; for if thou do this thing in Jupiter thou wilt blaspheme \cjRL{yhwh} and in Osiris \cjRL{yh/swh}. Ask and ye shall have! Seek, and ye shall find! Knock, and it shall be opened unto you!
