\addchap{One Star in Sight [Abridged]}

\begin{verse}
  Thy feet in mire, thine head in murk, \\
  \hspace{1em} O man, how piteous thy plight, \\
  The doubt that daunt, the ills that irk, \\
  \hspace{1em} Thou  hast nor wit nor will to fight --- \\
  How hope in heart, or worth in work? \\
  \hspace{1em} No star in sight!  

  Thy Gods proved puppets of the priest. \\
  \hspace{1em} \enquote{Truth? All's relation!} science sighed. \\
  In bondage with thy brother's beast, \\
  \hspace{1em} Love tortured thee, as Love's hope died \\
  And Love's faith rotted. Life no least \\
  \hspace{1em} Dim star descried. 

  Thy cringing carrion cowered and crawled \\
  \hspace{1em}To find itself a chance-cast clod \\
  Whose Pain was purposeless; appalled \\
  \hspace{1em}That aimless accident thus trod \\
  Its agony, that void skies sprawled \\
  \hspace{1em}On the vain sod! 

  All souls eternally exist, \\
  \hspace{1em}Each individual, ultimate, \\
  Perfect --- each makes itself a mist \\
  \hspace{1em}Of mind and flesh to celebrate \\
  With some twin mask their tender tryst \\
  \hspace{1em}Insatiate. 

  Some drunkards, doting on the dream, \\
  \hspace{1em}Despair that it should die, mistake \\
  Themselves for their own shadow-scheme. \\
  \hspace{1em}One star can summon them to wake \\
  To self; star-souls serene that gleam \\
  \hspace{1em}On life's calm lake. 

  That shall end never that began. \\
  \hspace{1em}All things endure because they are. \\
  Do what thon wilt, for every man \\
  \hspace{1em}And every woman is a star. \\
  Pan is not dead; he liveth, Pan! \\
  \hspace{1em}Break down the bar! 

  To man I come, the number of \\
  \hspace{1em}A man my number, Lion of Light; \\
  I am The Beast whose Law is Love. \\
  \hspace{1em}Love under will, his royal right --- \\
  Behold within, and not above, \\
  \hspace{1em}One star in sight! 
\end{verse}

\addsec*{One Star in Sight}

A glimpse of the structure and system of the Great White Brotherhood.

\begin{center}
  \Argentium{}\footnote{The Name of The Order and those of its three divisions are not disclosed to the profane. Certain swindlers have recently stolen the initials \Argentium{} in order to profit by its reputation.}.
\end{center}

Do what thou wilt shall be the whole of the Law.

1. The Order of the Star called S. S. is, in respect of its existence upon the Earth, an organised body of men and women distinguished among their fellows by the qualities here enumerated. They exist in their own Truth, which is both universal and unique.

They move in accordance with their own Wills, which are each unique, yet coherent with the universal will.

They perceive (that is, understand, know, and feel) in love, which is both unique and universal.

2. The order consists of eleven grades or degrees, and is numbered as follows: these compose three groups, the Orders of the S. S., of the R. C., and of the G. D. respectively.

\subsection*{The Order of the S. S.}
\begin{center}
  \begin{tabular}{p{.4\textwidth} r}
    Ipsissimus \dotfill & $10^{\circ}=1^{\square}$ \\
    Magus \dotfill & $9^{\circ}=2^{\square}$ \\
    Magister Templi \dotfill & $8^{\circ}=3^{\square}$ \\
    \end{tabular}
\end{center}

\subsection*{The Order of the R. C.}
\begin{center}
  (Babe of the Abyss --- the link)

  \begin{tabular}{p{.4\textwidth} r}
    Adeptus Exemptus \dotfill & $7^{\circ}=4^{\square}$ \\
    Adeptus Major \dotfill & $6^{\circ}=5^{\square}$ \\
    Adeptus Minor \dotfill & $5^{\circ}=6^{\square}$ \\
  \end{tabular}
\end{center}

\subsection*{The Order of the G. D.}
\begin{center}
  (Dominus Liminis --- the link)

  \begin{tabular}{p{.4\textwidth} r}
    Philosophus \dotfill & $4^{\circ}=7^{\square}$ \\
    Practicus \dotfill & $3^{\circ}=8^{\square}$ \\
    Zelator \dotfill & $2^{\circ}=9^{\square}$ \\
    Neophyte \dotfill & $1^{\circ}=10^{\square}$ \\
    Probationer \dotfill & $0^{\circ}=0^{\square}$ \\
  \end{tabular}
\end{center}

(These figures have special meanings to the initiated and are commonly employed to designate the grades.)

The general characteristics and attributions of these Grades are indicated by their correspondences on the Tree of Life, as may be studied in detail in the Book 777.

\textbf{Student.} --- His business is to acquire a general intellectual knowledge of all Systems of attainment, as declared in the prescribed books. \sout{(See curriculum in Appendix I.)}\editorsnote{A very long, useful list of readings, both within the \Argentium{} and without, has been omitted from this collection.}

\textbf{Probationer.} --- His principal business is to begin such practices as he may prefer, and to write a careful record of the same for one year.

\textbf{Neophyte.} --- Has to acquire perfect control of the Astral Plane.

\textbf{Zelator.} --- His main work is to achieve complete success in Asana and Pranayama. He also begins to study the formula of the Rosy Cross.

\textbf{Practicus.} --- Is expected to complete his intellectual training, and in particular to study the Qabalah.

\textbf{Philosophus.} --- Is expected to complete his moral training. He is tested in Devotion to the Order.

\textbf{Dominus Liminis.} --- Is expected to show mastery of Pratyahara and Dharana.

\textbf{Adeptus (without).} --- Is expected to perform the Great Work and to attain the Knowledge and Conversation of the Holy Guardian Angel.

\textbf{Adeptus (within).} --- Is admitted to the practice of the formula of the Rosy Cross on entering the College of the Holy Ghost.

\textbf{Adeptus (Major).} --- Obtains a general mastery of practical Magick, though without comprehension.

\textbf{Adeptus (Exemptus).} --- Completes in perfection all these matters. He then either (a) becomes a Brother of the Left Hand Path or, (b)  is stripped of all his attainments and of himself as well, even of his Holy Guardian Angel, and becomes a Babe of the Abyss, who, having transcended the Reason, does nothing but grow in the womb of its mother. It then finds itself a

\textbf{Magister Templi.} --- (Master of the Temple) : whose functions are fully described in Liber 418, as is this whole initiation from Adeptus Exemptus. See also \enquote{Aha!}. His principal business is to tend his \enquote{garden} of disciples, and to obtain a perfect understanding of the Universe. He is a Master of Samadhi.

\textbf{Magus.} --- Attains to wisdom, declares his law (See Liber I, vel Magi) and is a Master of all Magick in its greatest and
highest sense.

\textbf{Ipsissimus.} --- Is beyond all this and beyond all comprehension of those of lower degrees.

But of these last three Grades see some further account in The Temple of Solomon the King, Equinox I to X and elsewhere.

It should be stated that these Grades are not necessarily attained fully, and in strict consecution, or manifested wholly on all planes. The subject is very difficult, and entirely beyond the limits of this small treatise.

[The more detailed account is omitted from this text.]

Love is the Law, love under will.
