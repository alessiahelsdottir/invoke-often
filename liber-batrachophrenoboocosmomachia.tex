\addchap[Liber Batrachophrenoboocosmomachia]{Liber \textgreek{Βατραχοφρενοβοοκοσμομαχια}}\index{Grade Studies \& Work!Practicus (\grade{3}{8})}\index{Expansion of the Field of the Mind}\index{Grade Studies \& Work!Probationer (\grade{0}{0})}
\chapnum{DXXXVI\footnote{Refers to \cjRL{mslwt}\footnotemark the sphere of the Fixed Stars. From the Equinox Vol. 1, No. 10}}\footnotetext{This is as given in the Syllabus of the \Argentium{}; online sources say \cjRL{mzlwt} \textasciitilde{} \enquote{Mazloth}. --- \textsc{Editor}.}

\epigraph{Within His skull exist daily thirteen thousand myriads of Worlds, which draw their existence from Him, and by Him are upheld.}{I.R.Q. iii. 43.}

\textbf{Instruction in the expansion of the field of the mind.}

0. Let the Practicus study the textbooks of astronomy, travel, if need be, to a land where the sun and stars are visible, and observe the heavens with the best telescopes to which he may have access. Let him commit to memory the principal facts, and (at least roughly) the figures of the science.

1. Now, since these figures will leave no direct impression with any precision upon his mind, let him adopt this practice A.


\begin{quoting}

A. Let the Practicus be seated before a bare square table, and let an unknown number of small similar objects be thrown by his chela from time to time upon the table, and by that chela be hastily gathered up.

Let the Practicus declare at the glance, and the chela confirm by his count, the number of such objects.

The practice should be for a quarter of an hour thrice daily. The maximum number of objects should at first be seven. This maximum should increase by one at each practice, provided that not a single mistake is made by the Practicus in appreciating the number thrown.

This practice should continue assiduously for at least one year.

The quickness of the chela in gathering up the objects is expected to increase with time. The practice need not be limited to a quarter of an hour thrice daily after a time, but increased with discretion. Care must be taken to detect the first symptoms of fatigue, and to stop, if possible, even before it threatens. The practiced psychologist learns to recognise even minute hesitations that mark the forcing of the attention.

\end{quoting}


2. Alternating with the above, let the Practicus begin this practice B. It is assumed that he has thoroughly conquered the elementary difficulties of Dharana\index{Dharana}, and is able to prevent mental pictures from altering shape, size and colour against his will.


\begin{quoting}

B. Seated in the open air, let him endeavour to form a complete mental picture of himself and his immediate surroundings. It is important that he should be in the centre of such picture, and able to look freely in all directions. the finished picture should be a complete consciousness of the whole fixed, clear, and definite.

Let him gradually add to this picture by including objects more and more distant, until he have an image of the whole field of vision.

He will probably discover that it is very difficult to increase the apparent size of the picture as he proceeds, and it should be his most earnest endeavour to do so. he should seek in particular to appreciate distances, almost to the point of combating the laws of perspective.

\end{quoting}


3. These practices A and B accomplished, and his studies in astronomy completed, let him attempt this practice C.


\begin{quoting}

C. Let the Practicus form a mental picture of the Earth, in particular striving to realise the size of the Earth in comparison with himself, and let him not be content until by assiduity he has well succeeded.

Let him add the moon, keeping well in mind the relative sizes of, and the distance between, the planet and its satellite.

He will probably find the final trick of the mind to be a constant disappearance of the image, and the appearance of the same upon a smaller scale. This trick he must outwit by constancy of endeavour.

He will then in add in turn Venus, Mars, Mercury and the Sun.

It is permissible at this stage to change the point of view to the centre of the Sun, and to do so may add stability to the conception.

The Practicus may then add the Asteroids, Jupiter, Saturn, Uranus and Neptune. The utmost attention to detail is now necessary, as the picture is highly complex, apart from the difficulty of appreciating relative size and distance.

Let this picture be practiced month after month until it is absolutely perfect. The tendency which may manifest itself to pass into Dhyana\index{Dhyana} and Samadhi\index{Samadhi} must be resolutely combated with the whole strength of the mind.

Let the Practicus then re-commence the picture, starting from the Sun, and adding the planets one by one, each with its proper motion, until he have an image perfect in all respect of the Solar System as it actually exists. Let him particularly note that unless the apparent size approximate to the real, his practice is wasted. Let him then add a comet to the picture; he may find, perhaps that the path of this comet may assist him to expand the sphere of his mental vision until it include a star.

And thus, gathering one star after another, let his contemplation become vast as the heaven, in space and time ever aspiring to the perception of the Body of Nuit; yea, the Body of Nuit.
\end{quoting}
