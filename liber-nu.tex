\addchap{Liber Nu}
\chapnum{XI\footnote{Equinox Vol. 1, No. 7}}

\textbf{An instruction for attaining Nuit.}

000. This is the Book of the Cult of the Infinite Without.

00. The Aspirant is Hadit. Nuit is the infinite expansion of the Rose; Hadit is the infinite concentration of the Rood. \textit{(Instruction of V.V.V.V.V.)}

0. First let the Aspirant learn in his heart the First Chapter of the Book of the Law. \textit{(Instruction of V.V.V.V.V.)}

1. Worship, \textit{i.e.} identify thyself with, the Khabs, the secret Light within the Heart. Within this again, unextended, is Hadit.

\textit{This is the first practice of the Meditation (ccxx. I. 6 and 21).}

2. Adore and understand the Rim of the St\`{e}l\`{e} of Revealing.
\begin{quoting}[indentfirst=false]
\enquote{Above, the gemmed azure is \\
The naked splendor of Nuit; \\
She bends in ecstasy to kiss \\
The secret ardours of Hadit}
\end{quoting}

\textit{This is the first practice of Intelligence (ccxx. I. 14).}

3. Avoid any act of choice or discrimination.

\textit{This is the first practice of Ethics (ccxx. I. 22).}

4. Consider the six and fifty that 50/6=0.12. \begin{enumerate}[label={}]
\item o the circumference Nuit.
\item . the centre, Hadit.
\item 1 the unity proceeding, Ra-Hoor-Khuit.
\item 2 the world of illusion.
\item Nuit thus comprehends All in None.
\item Also $50 + 6 = 56 = 5 + 6 = 11$, the key to all Rituals.
\item And $50 \times 6 = 300$, the Spirit of the Child within.
\end{enumerate}

(Note N\textsubscript{\textgreek{Fις}}=72, the Shemhamphorash and the Quinaries of the Zodiac, etc.)

\textit{This is the second practice of Intelligence (ccxx. I. 24, 25).}

5. The Result of this Practice is the Consciousness of the Continuity of Existence, the Omnipresence of the Body of Nuit.

In other words, the Aspirant is conscious only of the Infinte Universe as a single Being. (Note for this the importance of Paragraph 3. \textsc{Ed [Equinox]}.)

\textit{This is the first Indication of the Nature of the Result (ccxx. I. 26).}

6. Meditate upon Nuit as the Continuous One resolved into None and Two as the phases of her being.

[For the Universe being self-contained must be capable of expression by the formula $(n-n)=0$. For if not, let it be expressed by the formula $n-m=p$. That is, the Infinite moves otherwise than within itself, which is absurd. \textsc{Ed [Equinox]}.]

\textit{This is the second practice of Meditation (ccxx. I. 27).}

7. Meditate upon the facts of Samadhi on all planes, the liberation of heat in chemistry, joy in natural history, Ananda in religion, when two things join to lose themselves in a third.

\textit{This is the third practice of Meditation (ccxx. I. 28, 29, 30).}

8. Let the Aspirant pay utmost reverence to the Authority of the \Argentium{} and follow Its instructions, and let him swear a great Oath of Devotion unto Nuit.

\textit{This is the second practice of Ethics (ccxx. I. 32).}

9. Let the Aspirant beware the slightest exercise of his will against another being. Thus, lying is a better posture than sitting or standing, as it opposes less resistance to gravitation. Yet his first duty is to the force nearest and most potent; \textit{e.g.} he may rise to greet a friend.

\textit{This is the third practice of Ethics (ccxx. I. 41).}

10. Let the Aspirant exercise his will without the least consideration for any other being. This direction cannot be understood, much less accomplished, until the previous practice has been perfected.

\textit{This is the fourth practice of Ethics (ccxx. I. 42, 43, 44).}

11. Let the Aspirant comprehend that these two practices are identical.

\textit{This is the third practice of Intelligence (ccxx, I, 45).}

12. Let the Aspirant live the Life Beautiful and Pleasant. For this freedom hath he won. But let each act, especially of love, be devoted wholly to his true mistress, Nuit.

\textit{This is the fifth practice of Ethics (ccxx. I. 51, 52, 61, 63).}

13. Let the Aspirant yearn toward Nuit under the stars of Night, with a love directed by his Magical Will, not merely proceeding from the heart.

\textit{This is the first practice of Magick Art (ccxx. I. 57).}

14. The Result of this Practice in the subsequent life of the Aspirant is to fill him with unimaginable joys: to give him certainty concerning the nature of the phenomenon called death, to give him peace unalterable, rest, and ecstasy.

\textit{This is the second Indication of the Nature of the Result (ccxx. I. 59).}

15. Let the Aspirant prepare a perfume of resinous woods and gums, according to his inspiration.

\textit{This is the second practice of Magick Art (ccxx. I. 59).}

16. Let the Aspirant prepare a Pantacle, as follows.

Inscribe a circle within a Pentagram, upon a ground square or of such other convenient shape as he may choose. Let the circle be scarlet, the Pentagram black, the ground royal blue studded with golden stars.

Within the circle, at its centre, shall be painted a sigil that shall be revealed to the Aspirant by Nuit herself.

And this Pentacle shall serve for a Telismatic Image, or as an Eidolon, or as a Focus for the mind.

\textit{This is the third practice of Magick Art (ccxx, I, 60).}

17. Let the Aspirant find a lonely place, if possible a place in the Desert of Sand, or if not, a place unfrequented, and without objects to disturb the view. Such are moorlands, fens, the open sea, broad rivers, and open fields. Also, and especially, the summits of mountains.

There let him invoke the Goddess as he hath Wisdom and Understanding so to do. But let this Invocation be that of a pure heart, \textit{i.e.} a heart wholly devoted to Her, and let him remember that it is Hadit Himself in the most secret place thereof that invoketh. Then let this serpent Hadit burst into flame.

\textit{This is the fourth practice of Magick Art (ccxx. I. 61).}

18. Then shall the Aspirant come a little to lie in Her bosom.

\textit{This is the third Indication of the Nature of the Result (ccxx. I. 61).}

19. Let the Aspirant stand upon the edge of a precipice in act or in imagination. And let him imagine and suffer the fear of falling.

Next let him imagine with this aid that the Earth is falling, and he with it, or he from it; and considering the infinity of space, let him excite the fear within him to the point of ecstasy, so that the most dreadful dream of falling that he hath ever suffered be as nothing in comparison.

\textit{This is the forth practice of Meditation. (Instruction of V.V.V.V.V.)}

20. Thus having understood the nature of this Third Indication, let him in his Magick Rite fall from himself into Nuit, or expand into Her, as his imagination may compel him.

And at that moment, desiring earnestly the Kiss of Nuit, let him give one particle of dust, \textit{i.e.}, let Hadit give himself up utterly to Her.

\textit{This is the fifth practice of Magick Art (ccxx. I. 61).}

21. Then shall he lose all in that hour.

\textit{This is the fourth Indication of the Nature of the Result (ccxx. I. 61).}

22. Let the Aspirant prepare a lovesong of rapture unto the Goddess, or let him be inspired by Her unto this.

\textit{This is the sixth practice of Magick Art (ccxx. [I.]\editorsnote{Missing a chapter indication in the Equinox. I have guessed it was also I.} 63).}

23. Let the Aspirant be clad in a single robe. An \enquote{abbai} of scarlet wrought with gold is most suitable. (An abbai is not unlike the Japanese kimono. It must fold simply over the breast without belt or other fastening. \textsc{Ed [Equinox]}.)

\textit{This is the seventh practice of Magick Art (ccxx. I. 61).}

24. Let the Aspirant wear a rich head-dress. A crown of gold adorned with sapphires or diamonds with a royal blue cap of maintenance, or nemmes, is most suitable.

\textit{This is the eighth practice of Magick Art (ccxx. I. 61).}

25. Let the Aspirant wear many jewels such as he may possess.

\textit{This is the ninth practice of Magick Art (ccxx. I. 61).}

26. Let the Aspirant prepare an Elixir or libation as he may have wit to do.

\textit{This is the tenth practice of Magick Art (ccxx. I. 63).}

27. Let the Aspirant invoke, lying supine, his robe spread out as it were a carpet.

\textit{This is the eleventh practice of Magick Art. (Instruction of V.V.V.V.V.)}

28. Summary. Preliminaries. These are the necessary possessions.
\begin{enumerate}[leftmargin=4\parindent]
\item The Crown or head-dress.
\item The Jewels.
\item The Pantacle.
\item The Robe.
\item The Song or Incantation.
\item The Place of Invocation.
\item The Perfume.
\item The Elixir.
\end{enumerate}

29. Summary continued. Preliminaries. These are the necessary comprehensions.

\begin{enumerate}[leftmargin=4\parindent]
\item The Natures of Nuit and Hadit, and their relation.
\item The Mystery of the Individual Will.
\end{enumerate}

30. Summary continued. Preliminaries. These are the meditations necessary to be accomplished.
\begin{enumerate}[leftmargin=4\parindent]
\item The discovery of Hadit in the Aspirant, and identification with Him.
\item The Continuous One.
\item The Value of the Equation $n + (-n)$.
\item Cremnophobia.
\end{enumerate}

31. Summary continued. Preliminaries. These are the Ethical Practices to be accomplished.
\begin{enumerate}[leftmargin=4\parindent]
\item Assertion of the Kether-point-of-view.
\item Reverence to the Order.
\item Abolition of human will.
\item Exercise of true will.
\item Devotion to Nuit throughout a beautified life.
\end{enumerate}

32. Summary continued. The Actual Rite.
\begin{enumerate}[leftmargin=4\parindent]
\item Retire to desert with crown and other insignia and implements.
\item Burn perfume.
\item Chant incantation.
\item Drink unto Nuit the Elixir.
\item Lying supine, with eyes fixed on the stars, practice the sensation of falling into nothingness.
\item Being actually within the bosom of Nuit, let Hadit surrender Himself.
\end{enumerate}

33. Summary concluded. The Results.
\begin{enumerate}[leftmargin=4\parindent]
\item Expansion of consciousness to that of the Infinte.
\item \enquote{Loss of all} the highest mystical attainment.
\item True Wisdom and perfect Happiness.
\end{enumerate}

