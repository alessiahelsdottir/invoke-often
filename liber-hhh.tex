{
\raggedbottom
\addchap{Liber HHH}\index{Grade Studies \& Work!Neophyte (\grade{1}{10})}
\chapnum{CCCXLI\footnote{The Sum of the 3 Mothers of the Alphabet. From the Equinox Vol. 1, No. 5; Magick in Theory and Practice.}}

{

\begin{sloppypar}
`Sunt duo modi per quos homo fit Deus: Tohu et Bohu.

`Mens quasi flamma sugat, aut quasi puteus aquae quiescat.

`Alteri modi sunt tres exempli, qui illis extra limine collegii sancti dati sunt.

`In hoc primo libro sunt Aquae Contemplationis.'

\textit{Two are the methods of becoming God: the Upright and the Averse.}

\textit{Let the Mind become as a flame, or as a well of still water.}

\textit{Of each method are three principal examples given to them that are without the Threshold.}

\textit{In this first book are written the Reflexions.} \\

`Sunt tres contemplationes quasi halitus in mente humana abysso inferni. Prima, \textgreek{Νεκρος}; secunda, \textgreek{Πυραμις}; tertia, \textgreek{Φαλλος} vocatur. Et hae reflexiones aquaticae sunt trium enthusiasmorum, Apollonis, Dionysi, Veneris.

`Tota stella est Nechesh et Messiach, nomen \cjRL{'hyh} cum \cjRL{yhyh} conjunctum.'

\textit{There are three contemplations as it were breaths in the human mind, that is the Abyss of Hell: the first is called }\textgreek{Νεκρος}\textit{ the second }\textgreek{Πυραμις}\textit{, and the third }\textgreek{Φαλλος}.\textit{ These are the watery reflexions of the three enthusiasms; those of Apollo, Dionysus, and Aphrodite.}

\textit{The whole star is Nechesh and Messiach, the name \cjRL{'hyh}{} joined with \cjRL{yhyh}{}.}
\end{sloppypar}

}
}

\pagebreak
\addsec*{Three methods of attainment.}

\addsec*{CONTINET CAPITULA TRIA: MMM, AAA, ET SSS.}

\addsec*{I.}

\addsec*{MMM}\index{Grade Studies \& Work!Zelator (\grade{2}{9})}

\epigraph{\enquote{I remember a certain holy day in the dusk of the Year, in the dusk of the Equinox of Osiris, when first I beheld thee visibly; when first the dreadful issue was fought out; when the Ibis-headed One charmed away the strife. I remember thy kiss, even as a maiden should. Nor in the dark byways was there another: thy kisses abide.}}{\textsc{Liber Lapidis Lazuli. VII. 15. 16.}}


0. Be seated in thine Asana, wearing the robe of a Neophyte, the hood drawn.

1. It is night, heavy and hot, there are no stars. Not one breath of wind stirs the surface of the sea, that is thou. No fish play in thy depths.

2. Let a Breath rise and ruffle the waters. This also thou shalt feel playing upon thy skin. It will disturb thy meditation twice or thrice, after which thou shouldst have conquered this distraction. But unless thou first feel it, that Breath hath not arisen.

3. Next, the night is riven by the lightning flash. This also shalt thou feel in thy body, which shall shiver and leap which the shock, and that also must both be suffered and overcome.

4. After the lightning flash, resteth in the zenith a minute point of light. And that light shall radiate until a right cone be established upon the sea, and it is day.

With this thy body shall be rigid, automatically; and this shalt thou endure, withdrawing thyself into thine heart in the form of an upright Egg of blackness; and therein shalt thou abide for a space.

5. When all this is perfectly and easily performed at will, let the aspirant figure to himself a struggle with the whole force of the Universe. In this he is only saved by his minuteness. But in the end he is overcome by Death, who covers him with a black cross.

Let his body fall supine with arms outstretched.

6. So lying, let him aspire fervently unto the Holy Guardian Angel\index{Holy Guardian Angel}.

7. Now let him resume his former posture.

Two and twenty times shall he figure to himself that he is bitten by a serpent, feeling even in his body the poison thereof, And let each bite be healed by an eagle or hawk, spreading its wings above his head, and dropping thereupon a healing dew. But let the last bite be so terrible a pang at the nape of the neck that he seemeth to die, and let the healing dew be of such virtue that he leapeth to his feet.

8. Let there be now placed within his egg a red cross, then a green cross, then a golden cross, then a silver cross; or those things which these shadow forth. Herein is silence; for he that hath rightly performed the meditation will understand the inner meaning hereof, and it shall serve as a test of himself and his fellows.

9. Let him now remain in the Pyramid or Cone of Light, as an Egg, but no more of blackness.

10. Then let his body be in the position of the Hanged Man, and let him aspire with all his force unto the Holy Guardian Angel\index{Holy Guardian Angel}.

11. The grace having been granted unto him, let him partake mystically of the Eucharist of the Five Elements and let him proclaim Light in Extension; yea, let him proclaim Light in Extension.


\pagebreak

\addsec*{II.}\index{Grade Studies \& Work!Zelator (\grade{2}{9})}

\addsec*{AAA}

\epigraph{\enquote{These loosen the swathings of the corpse; these unbind the feet of Osiris, so that the flaming God may rage through the firmament with his fantastic spear.}}{\textsc{Liber Lapidis Lazuli. VII. III.}}

0. Be seated in thine Asana, or recumbent in Shavasana, or in the position of the dying Buddha.

1. Think of thy death; imagine the various diseases that may attack thee, or accidents overtake thee. Picture the process of death, applying always to thyself.

(A useful preliminary practice is to read textbooks of Pathology, and to visit museums and dissecting-rooms.)

2. Continue this practice until death is complete; follow the corpse through the stages of embalming, wrapping and burial.

3. Now imagine a divine breath entering thy nostrils.

4. Next, imagine a divine light enlightening the eyes.

5. Next, imagine the divine voice awakening the ears.

6. Next, imagine a divine kiss imprinted on the lips.

7. Next, imagine the divine energy informing the nerves and muscles of the body, and concentrate on the phenomenon which will already have been observed in 3, the restoring of the circulation.

8. Last, imagine the return of the reproductive power, and employ this to the impregnation of the Egg of light in which man is bathed.

9. Now represent to thyself that this Egg is the Disk of the Sun, setting in the west.

10. Let it sink into blackness, borne in the bark of heaven, upon the back of the holy cow Hathor. And it may be that thou shalt hear the moaning thereof.

11. Let it become blacker than all blackness. And in this meditation thou shalt be utterly without fear, for that the blackness that will appear unto thee is a thing dreadful beyond all thy comprehension.

And it shall come to pass that if thou hast well and properly performed this meditation that on a sudden thou shalt hear the drone and booming of a Beetle.

12. Now then shall the Blackness pass, and with rose and gold shalt thou arise in the East with the cry of an Hawk resounding in thine ear. Shrill shall it be and harsh.

13. At the end shalt thou rise and stand in the mid-heaven, a globe of glory. And therewith shall arise the might Sound that holy men have likened unto the roaring of a Lion.

14. Then shalt thou withdraw thyself from the Vision, gathering thyself into the divine form of Osiris upon his throne.

15. Then shalt thou repeat audibly the cry of triumph of the god re-arisen, as it shall have been given unto thee by thy Superior.

16. And this being accomplished, thou mayest enter again into the Vision, that thereby shall be perfected in Thee.

17. After this shalt thou return into the Body, and give thanks unto the Most High God IAIDA, yea unto the Most High God IAIDA.

18. Mark well that this operation should be performed if it be possible in a place set apart and consecrated to the Works of the Magick of Light. Also that the Temple should be ceremonially open as thou hast knowledge and skill to perform, and that at the end thereof the closing should be most carefully accomplished. But in the preliminary practice it is enough to cleanse thyself by ablution, by robing, and by the rituals of the Pentagram and Hexagram.

0-2 should always be practiced at first, until some realisation is obtained; and the practice should always be followed by a divine invocation of Apollo or of Isis or of Jupiter or of Serapis.

Next after a swift summary of 0-2 practice 3-7.

This being mastered, add 8.

Then add 9-13.

Then being prepared and fortified, well fitted for the work, perform the whole meditation at one time. And let this be continued until perfect success be attained therein. For this is a mighty meditation and holy, having power even upon Death, yea, having power even upon Death.

(Note by Fra. O. M. At any time during this meditation the concentration may bring about Samadhi. This is to be feared and shunned, more than any other breaking of control, for that it is the most tremendous of the forces which threaten to obsess. There is also some danger of acute delirious melancholia at point 1.)


\pagebreak

\addsec*{III.}

\addsec*{SSS}\index{Grade Studies \& Work!Practicus (\grade{3}{8})}
\epigraph{
`Thou art a beautiful thing, whiter than a woman in the column of this vibration. \\
`I shoot up vertically like an arrow, and become that Above. \\
`But it is death, and the flame of the pyre. \\
`Ascend in the flame of the pyre, o my Soul! Thy God is like the cold emptiness of the utmost heaven, into which thou radiatest thy little light. \\
`When Thou shalt know me, O empty God, my flame shall utterly expire in Thy great N.\nolinebreak{} O.\nolinebreak{} X.\nolinebreak{} ' }{\textsc{Liber Lapidis Lazuli. I. 36-40.}}


0. Be seated in thine Asana, preferably the Thunderbolt.

It is essential that the spine be vertical.

1. In this practice the cavity of the brain is the Yoni; the spinal cord the Lingam.

2. Concentrate thy thought of adoration in the brain.

3. Now begin to awaken the spine in this manner. Concentrate thy thought of thyself in the base of the spine, and move it gradually up a little at a time.

By this means thou wilt become conscious of the spine, feeling each vertebra as a separate entity. This must be achieved most fully and perfectly before the further practice is begun.

4. Next, adore the brain as before, but figure to thyself its content as infinite. Deem it to be the womb of Isis, or the body of Nuit.

5. Next, identify thyself with the base of the spine as before, but figure to thyself its energy as infinite. Deem it to be the phallus of Osiris or the being of Hadit.

6. These two concentrations 4 and 5 may be pushed to the point of Samadhi. Yet lose not control of the will; let not Samadhi be thy master herein.

7. Now then, being conscious both of the brain and the spine, and unconscious of all else, do thou imagine the hunger of the one for the other; the emptiness of the brain, the ache of the spine, even as the emptiness of space and the aimlessness of Matter.

And if thou hast experience of the Eucharist in both kinds, it shall aid thine imagination herein.

8. Let this agony grow until it be insupportable, resisting by will every temptation. Not until thine whole body is bathed in sweat, or it may be in sweat of blood, and until a cry of intolerable anguish is forced from thy closed lips, shalt thou proceed.

9. Now let a current of light, deep azure flecked with scarlet, pass up and down the spine, striking as it were upon thyself that art coiled at the base as a serpent.

Let this be exceeding slow and subtle; and though it be accompanied with pleasure, resist; and though it be accompanied with pain, resist.

10. This shalt thou continue until thou art exhausted, never relaxing the control. Until thou canst perform this one section during a whole hour, proceed not. And withdraw from the meditation by an act of will, passing into a gentle Pranayama without Kumbhakham\index{Kumbhaka}, and meditating on Harpocrates, the silent and virginal God.

11. Then at last, being well-fitted in body and mind, fixed in peace, beneath a favourable heaven of stars, at night, in calm and warm weather, mayst thou quicken the movement of the light until it be taken up by the brain and the spine, independently of thy will.

12. If in this hour thou shouldst die, is it not written, \enquote{Blessed are the dead that die in the Lord}? Yea, Blessed are the dead that die in the Lord!
