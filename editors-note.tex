{
\begin{center}
\Large
\textbf{MAGICK}

is for

\textbf{ALL}
\end{center}


\begin{center}
  \rule{1in}{0.5pt}
\end{center}

\enquote{Frater Perdurabo is the most honest of all the great religious teachers. Others have said: \enquote{Believe me!} He says: \enquote{Don't believe me!}

[\dots]

Those who have wished men to believe in them were absurd. A persuasive tongue or pen, or an efficient sword, with rack and stake, produced this \enquote{belief,} which is contrary to, and destructive of, all real religious experience.

 The whole life of Frater Perdurabo is now devoted to seeing that you obtain this living experience of Truth for, by, and in yourselves!}

\textit{\textemdash{} Soror Virakam (Mary d'Este Sturges), Book Four (Mysticism).}

\begin{center}
  \rule{1in}{0.5pt}
\end{center}

\enquote{In this book it is spoken of the Sephiroth and the Paths; of Spirits and Conjurations; of Gods, Spheres, Planes, and many other things which may or may not exist.

It is immaterial whether these exist or not. By doing certain things certain results will follow; students are most earnestly warned against attributing objective reality or philosophic validity to any of them.}

\textit{\textemdash{} Frater Perdurabo (Aleister Crowley), Liber O.}

\begin{center}
  \rule{1in}{0.5pt}
\end{center}

\enquote{Why should you study and practice Magick? Because you can't help doing it, and you had better do it well than badly. You are on the links, whether you like it or not; why go on topping your drive, and slicing your brassie, and fluffing your niblick, and pulling your iron, and socketing your mashie and not being up with your putt\textemdash{}that's 6, and you are not allowed to pick up. It's a far cry to the Nineteenth, and the sky threatens storm before the imminent night.}

\textit{\textemdash{} Frater Perdurabo, Magick Without Tears.}

\begin{center}
  \rule{1in}{0.5pt}
\end{center}
\vspace*{\fill}
\raggedbottom
\pagebreak
}
{
\vspace*{\fill}
\begin{center}
\huge
\textbf{Acknowledgements}
\end{center}

A heartfelt thanks to the most excellent creator of the website Keep Silence, who has been scanning and uploading original works of Aleister Crowley, at personal cost, and making them available to the world.

The scans this was transcribed from are available online at

\begin{center}
\url{https://keepsilence.org/the-equinox}
\end{center}

And a thanks to all those who have continued, expanded, and kept alive the work and spirit of the Great Beast \textemdash{} Israel Regardie, Nema, Kenneth Grant, Lon Milo DuQuette, Frater Achad, and innumerable others over the century since.

\vspace*{\fill}
}

\addchap*{Note from the Editor}

\textit{Do what thou wilt shall be the whole of the Law.}


It is my Will that there be a simple, mostly self-contained collection of Aleister Crowley's rituals and instructions as informed by his publications up to the 1929 publication of Magick in Theory and Practice. I therefore took my magickal \enquote{weapons} (old books and my laptop) and wrote innumerable \enquote{incantations} (the typesetting of this book) in a \enquote{magickal language} (\enquote{\LaTeX}), that allow me to transmute (\enquote{\textit{compile}}) it into a form that the spirits (\enquote{\textit{printers}}) understand that they may deliver unto me the collection in physical form.

The composition and distribution of this book is thus an act of Magick by which I cause Changes to take place in conformity with my Will.

This was compiled manually from the works of Aleister Crowley in the public domain for the Editor's personal use, but has been further edited and released with hope of being a portable\footnote{That adorations, instructions, and rituals may easily be referred to anywhere and everywhere.}, affordable\footnote{Free\footnotemark, digitally. Print it out if you will. Share it far and wide! (with attribution)}\footnotetext{Creative Commons Attribution-ShareAlike 4.0 International License}, high quality\footnote{Please contact me at invokeoften@protonmail.com with any errors and they will be fixed.\footnotemark}\footnotetext{Please note that there are different types of error and it is helpful to identify which has been found, \textit{e.g.} errors in transcription, and errors in the original works of Aleister Crowley.} collection of essential instructions the Editor has not otherwise been able to find\footnote{Gems of the Equinox (and Diamonds), as well as Magick in Theory and Practice, are more encyclop\ae{}dic in nature; and, clocking in at nearly 1000 pages each, not something you want to take with you and refer to quickly, usually. This is not intended to replace these great works from e.g. Weiser, but provide another form to work from.}.

In this text, there are minor changes to standardise on one spelling of a word (\textit{e.g.} \enquote{practice}), to use the Oxford comma, to change \enquote{cakkra} to the more modern \enquote{chakra}, and similar minor points of taste. In any case where magical formul\ae{} or quotations from Class A publications are represented, the intention is that not so much as a letter is changed. Paragraph numbers, line numbers, and so on have been maintained as in the originals.

This collection is divided into three sections:

\begin{enumerate}[label=\greek*]
\item \textbf{\textit{Instructions}} \textemdash{} the techniques which form the foundation of magickal practice. The disciplines of control of one's body (asana), breath (pranayama), and mind (dharana). Elementary techniques of Magick such as Rising on the Planes and assumption of God-forms. Magickal memory and past life recall.
\item \textbf{\textit{Rituals}} \textemdash{} the essential rituals and practices to attune with the \AE{}on of Horus.
\item \textbf{\textit{Attainment}} \textemdash{} techniques of the Great Work\footnote{The Great Work is identified with the realisation of one's True Self in Liber CL vel \cjRL{n`l} A Sandal de Lege Libellum L\textendash{}L\textendash{}L\textendash{}L\textendash{}L.}. Union with deities, dissolution of consciousness in the All, and invocations towards Knowledge and Conversation with one's Higher Self\footnote{Commonly called \enquote{Holy Guardian Angel} in Thelemic circles, but it has various names in various cultures and systems e.g. \enquote{Genius} (\GD{}), \enquote{Atman} (Hinduism), \enquote{Logos} (Gnosticism), \enquote{Augoeides} (Neo-Platonism), \enquote{Daemon} (Greeks, Jung), etc.}.
\end{enumerate}

Earlier chapters prepare for later ones where possible, so you may work from the beginning to end, mostly. E.g., Liber E gives an outline of pranayama, and Liber RV, later, further instructs in pranayama; Liber Turris vel Domus Dei suggests reading Liber E and Liber HHH first, so both come before it; and so on.

At the beginning of each chapter is the official document number of the text, and a footnote about where it was transcribed from, to allow for ease of sourcing for comparison.

The Editor does not in any way condone the more extreme practices suggested by Aleister Crowley such as in Liber Jugorum; alternatives are suggested as footnotes on occasion, but commentary from the Editor is mostly kept to a minimum. The lack of an alternative noted does not imply endorsement of the practice.

\textit{Love is the law, love under will.}


\addchap*{Description of the Selected Publications}


This is primarily \enquote{Class D} publications, Official Rituals and Instructions, along with some \enquote{Class B} publications, \enquote{consists of books or essays which are the result of ordinary scholarship, enlightened and earnest.}

A list of \enquote{Class D} publications was in the syllabus given in \textit{The Equinox} Vol. 1, No. 10, along with descriptions:

\begin{itemize}
\item Liber III. \textemdash{} \textit{Liber Jugorum.} An instruction for the control of speech, action, and thought.
\item Liber VIII. \textit{See}{}\-\ CCCCXVIII\footnote{The Vision and the Voice}. [Instructions to attain Knowledge and Conversation of one's Holy Guardian Angel.]
\item Liber XI. \textemdash{} \textit{Liber N V.} An instruction for attaining Nuit.
\item Liber XIII. \textemdash{} \textit{Graduum Montis Abiegni.} An account of the task of the Aspirant from Probationer to Adept.
\item Liber XVI. \textemdash{} \textit{Liber Turris Vel Domus Dei.} An instruction for the attainment by the direct destruction of thoughts as they arise in the mind.
\item \sout{Liber XVII. \textemdash{} \textit{Liber I A O.}} [Unpublished]
\item Liber XXV. \textemdash{} \textit{The Star Ruby.} An improved form of the \enquote{lesser} ritual of the Pentagram.
\item \sout{Liber XXVIII. \textemdash{} \textit{Liber Septem Regum Sanctorum.}} [Unpublished]
\item Liber XXXVI. \textemdash{} \textit{The Star Sapphire.} An improved ritual of the Hexagram.
\item Liber XLIV. \textemdash{} \textit{The Mass of the Phoenix.} An instruction for a simple and exoteric form of Eucharist.\footnote{Note in Liber ABA: \enquote{A Ritual of the Law}}
\item \sout{Liber C. \textemdash{} \textit{\cjRL{pk}}} [Unpublished]
\item \sout{Liber CXX. \textemdash{} \textit{Liber Cadaveris.} The Ritual of Initiation of a Zelator.} [Listed as unpublished in Magick, 1929]
\item Liber CLXXV. \textemdash{} \textit{Astarte Vel Liber Berylli.} An instruction in attainment by the method of devotion.
\item \sout{Liber CLXXXV. \textemdash{} \textit{Liber Collegii Sancti.} Being the tasks of the Grades and their Oaths proper to Liber XIII. This is the official Paper of the various grades. It includes the Task and Oath of a Probationer.} [Listed as unpublished in Magick, 1929]
\item Liber CC. \textemdash{} \textit{Resh Vel Helios.} An instruction for adorations of the Sun four times daily, with the object of composing the mind to meditation and of regularising the practices.
\item Liber CCVI. \textemdash{} \textit{Liber R V Vel Spiritus.} Full instruction in Pranayama.
\item Liber CCCLXI. \textemdash{} \textit{Liber H H H.} Gives three methods of attainment through a willed series of thoughts.
\item Liber CCCCXII. \textemdash{} \textit{A Vel Armorum.} An instruction for the preparation of the Elemental Instruments.
\item \sout{Liber CDLI. \textemdash{} \textit{Liber Siloam.}} [Unpublished]
\item Liber DLV. \textemdash{} \textit{Liber H A D.} An instruction for attaining Hadit.
\item \sout{Liber DCLXXI. \textemdash{} \textit{Liber Pyramidos.} The ritual of initiation of a Neophyte. It includes sub-rituals numbered from 672 to 676.} [Class D in the Equinox, but unpublished; later unlisted altogether in Magick, 1929]
\item Liber DCCCXXXI. \textemdash{} \textit{Liber I O D}, formerly called \textit{\textsc{Vesta}} An instruction giving three methods of reducing the manifold consciousness to Unity.
\item \sout{Liber \-\ \-\ \-\ \-\ \textemdash{} \textit{Liber Collegii Interni.}} [Unpublished]
\end{itemize}

In Magick, 1929, the following were also \enquote{Class D}:
\begin{itemize}
\item Liber DCCC. \textemdash{} \textit{\textbf{Liber Samekh} Theurgia Goetia Summa} (CONGRESSUS CUM DAEMONE) being the Ritual employed by the Beast 666 for the Attainment of the Knowledge and Conversation of his Holy Guardian Angel during the Semester of His performance of the Operation of the Sacred Magick of ABRAMELIN THE MAGE.
\item Liber V vel Reguli \textemdash{} Being the Ritual of the Mark of the Beast: an incantation proper to invoke the Energies of the \AE{}on of Horus, adapted for the daily use of the Magician of whatever grade.
\end{itemize}

Also included as explanatory of or as expansions on the above: \begin{itemize}
\item One Star In Sight\footnote{Abridged. Included as an expansion of the instructions in Graduum Montis Abiegni, but the longer essay on the grades is left out.} \textemdash{} An essay on the structure and system of the Great White Brotherhood\footnote{\Argentium{}}.
\item Liber O\footnote{Abridged. Missing the Greater Rituals of the Pentagram and Hexagram due to the non-instruction forms they are given in. Interpretations are readily available online.} vel Manus Et Sagitt\ae{} sub figura VI \textemdash{} Class \enquote{B}; \enquote{The instructions given in this book are too loose to find place in the Class D publications.} \textemdash{} Instructions given for elementary study of the Qabalah, Assumption of God forms, Vibration of Divine Names, the Rituals of Pentagram and Hexagram, and their uses in protection and invocation, a method of attaining astral visions so-called, and an instruction in the practice called Rising on the Planes.
\item Liber E vel Exercitiorum sub figura IX \textemdash{} Class \enquote{B} \textemdash{} Instructs the aspirant in the necessity of keeping a record. Suggests methods of testing physical clairvoyance. Gives instruction in Asana, Pranayama and Dharana, and advises the application of tests to the physical body, in order that the student may thoroughly understand his own limitations.
\item Liber DXXXVI. Βατραχοφρενοβοοκοσμομαχια \textemdash{} Class \enquote{B} \textemdash{} An instruction in expansion of the field of the mind.
\item Liber CMXIII. \cjRL{ty/s'rb}{} vel Thisharb vi\ae{}{} memori\ae{}{} \textemdash{} Class \enquote{B} \textemdash{} Gives methods for attaining the magical memory, or memory of past lives, and an insight into the function of the Aspirant in this present life.
\item Liber CDLXXIV. Os Abysmi vel Daath \textemdash{} Class \enquote{C} \textemdash{} An instruction in a purely intellectual method of entering the Abyss.
\end{itemize}
