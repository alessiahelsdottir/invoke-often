  \addchap{The Holy Hexagram}
  \addsec*{ΚΕΦΑΛΗ ΞΘ}
  \addsec*{The Way To Succeed \textemdash{} And The Way To Suck Eggs!}
  \begin{Verse}
  This is the Holy Hexagram. \\
  Plunge from the height, O God, and interlock with Man! \\
  Plunge from the height, O Man, and interlock with Beast! \\
  The Red Triangle is the descending tongue of grace; the Blue Triangle is the ascending tongue of prayer. \\
  This Interchange, the Double Gift of Tongues, the Word of Double Power \textemdash{} \textsc{Abrahadabra!} \textemdash{} is the sign of the GREAT WORK, for the GREAT WORK is accomplished in Silence. And behold is not that Word equal to Cheth, that is Cancer, whose Sigil is \cancer{}? \\
  This Work also eats up itself, accomplishes its own end, nourishes the worker, leaves no seed, is perfect in itself. \\
  Little children, love one another! \\
  \end{Verse}
  \pagebreak

  \addchap*{}
  \addsec*{}
  \addsec*{Commentary}

  The key to the understanding of this chapter is given in the number and the title, the former being intelligible to all nations who employ Arabic figures, the latter only to experts in deciphering English puns.

  The chapter alludes to Levi's drawing of the Hexagram, and is a criticism of, or improvement upon, it. In the ordinary Hexagram, the Hexagram of nature, the red triangle is upwards, like fire, and the blue triangle downwards, like water.  In the magical hexagram this is revered; the descending red triangle is that of Horus, a sign specially revealed by him personally, at the Equinox of the Gods.  (It is the flame descending upon the altar, and licking up the burnt offering.)  The blue triangle represents the aspiration, since blue is the colour of devotion, and the triangle, kinetically considered, is the symbol of directed force.

  In the first three paragraphs this formation of the hexagram is explained; it is a symbol of the mutual separation of the Holy Guardian Angel and his client. In the interlocking is indicated the completion of the work.

  \textemdash{} The Book of Lies.

  \vspace*{\fill}

