{
\raggedbottom
\addchap{Graduum Montis Abiegni}
\chapnum{XIII\footnote{Equinox Vol. 1, No. 3.}}

\addsec*{A Syllabus of the Steps Upon the Path}

\begin{quoting}
\small
51. Let not the failure and the pain turn aside the worshippers. The
foundations of the pyramid were hewn in the living rock ere sunset; did the
king weep at dawn that the crown of the pyramid was yet unquarried in the
distant land?

52. There was also an humming-bird that spake unto the horned cerastes,
and prayed him for poison. And the great snake of Khem the Holy One, the
royal Ur\ae{}us serpent, answered him and said:

53. I sailed over the sky of Nu in the car called Millions-of-Years, and I saw
not any creature upon Seb that was equal to me. The venom of my fang is
the inheritance of my father, and of my father’s father; and how shall I give
it unto thee? Live thou and thy children as I and my fathers have lived, even
unto an hundred millions of generations, and it may be that the mercy of the
Mighty Ones may bestow upon thy children a drop of the poison of eld.

54. Then the humming-bird was afflicted in his spirit, and he flew unto the flowers, and it was as if nought had been spoken between them. Yet in a little while, a serpent struck him that he died.

55. But an Ibis that meditated upon the bank of Nile the beautiful god
listened and heard. And he laid aside his Ibis ways, and became as a serpent
saying Peradventure in an hundred millions of millions of generations of my
children, they shall attain to a drop of the poison of the fang of the Exalted
One.

56. And behold! ere the moon waxed thrice he became an Ur\ae{}us serpent, and the poison of the fang was established in him and his seed even for ever and for ever.

--- \textsc{Liber LXV. Cap. V.} vv. 52-56.
\end{quoting}
}

\pagebreak

1. \textit{The Probationer.} His duties are laid down in Paper A, Class D. Being \textit{without}, they are vague and general. He receives Liber LXI. and LXV. (Certain Probationers are admitted after six months or more to Ritual XXVIII.)
\begin{itemize}
\item At the end of the Probation he passes Ritual DCLXXI., which constitutes him a Neophyte.
\end{itemize}

2. \textit{The Neophyte.} His duties are laid down in Paper B, Class D. He receives Liber VII.
\begin{itemize}
\item Examination in Liber O, Caps. I.-IV., Theoretical and Practical.
\item Examination in The Four Powers of the Sphinx. Practical.
\item Four tests are set.
\item Further, he builds up the magic Pentacle.
\item Finally he passes Ritual CXX, which constitutes him a Zelator.
\end{itemize}

3. \textit{The Zelator.} His duties are laid down in Paper C, Class D. He receives Liber CCXX., XXVII., and DCCCXIII.
\begin{itemize}
\item Examination in Posture and Control of Breath (see Equinox No. 1). Practical.
\item Further, he is given two meditation-practises corresponding to the two rituals DCLXXI and CXX.
\item (Examination is only in the knowledge of, and some little practical acquaintance with, these meditations. The complete results, if attained, would confer a much higher grade.)
\item Further, he forges the magic Sword.
\item No ritual admits to the grade of Practicus, which is conferred by authority when the task of the Zelator is accomplished.
\end{itemize}

4. \textit{The Practicus.} His duties are laid down in Paper D, Class D.
\begin{itemize}
\item Instruction and Examination in the Qabalah and Liber DCCLXXVII.
\item Instruction in Philosophical Meditation (Gnana-Yoga).\footnote{All these instructions will be issued openly in the Equinox in due course, where this has not already been done}
\item Examination in some one mode of divination: \textit{e.g.}, Geomancy, Astrology, the Tarot. Theoretical. He is given a meditation-practise on Expansion of Consciousness.
\item He is given a meditation-practise in the destruction of thoughts.
\item Instruction and Examination in Control of Speech. Practical.
\item Further, he casts the magic Cup.
\item No ritual admits to the grade of Philosophus, which is conferred by authority when the Task of the Practicus is accomplished.
\end{itemize}

5. \textit{The Philosophus.} His duties are laid down in Paper E, Class D.
\begin{itemize}
\item He practises Devotion to the Order.
\item Instruction and Examination in Methods of Meditation by Devotion (Bhakti-Yoga).
\item Instruction and Examination in Construction and Consecration of Talismans, and in Evocation.
\item Theoretical and Practical.
\item Examination in Rising on the Planes (Liber O, caps. V., VI.). Practical.
\item He is given a meditation-practise on the Senses, and the Sheaths of the Self, and the Practise called Mahasatipatthana.
\item (See The Sword of Song, \enquote{Science and Buddhism}.)
\item Instruction and Examination in Control of Action.
\item Further, he cuts the Magic Wand.
\item Finally, the Title of Dominus Liminis is conferred upon him.
\item He is given meditation-practises on the Control of Thought, and is instructed in Raja-Yoga.
\item He receives Liber Mysteriorum and obtains a perfect understanding of the Formul\ae{} of Initiation.
\item He meditates upon the diverse knowledge and power that he has acquired, and harmonises it perfectly.
\item Further, he lights the Magic Lamp.
\item At last, Ritual VIII. admits him to the grade of Adeptus Minor.
\end{itemize}

6. \textit{The Adeptus Minor.} His duty is laid down in Paper F, Class D.
\begin{itemize}[label={}]
\item It is to follow out the instruction given in the Vision of the Eighth \AEthyr{} for the attainment of the Knowledge and Conversation of the Holy Guardian Angel.
\end{itemize}

(NOTE. This is in truth the sole task; the others are useful only as adjuvants to and preparations for the One Work.

Moreover, once this task has been accomplished, there is no more need of human help or instruction; for by this alone may the highest attainment be reached.

All these grades are indeed but convenient landmarks, not necessarily significant. A person who had attained them all might be immeasurably the inferior of one who had attained none of them; it is Spiritual Experience alone that counts in Result; the rest is but Method.

Yet it is important to possess knowledge and power, provided that it be devoted wholly to that One Work.)
