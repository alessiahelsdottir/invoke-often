\addchap[Liber Resh vel Helios]{Liber \cjRL{r|}{} vel Helios}
\chapnum{CC\footnote{Equinox Vol. 1, No. 7; Magick in Theory and Practice.}}

\textbf{An instruction for the adoration of the Sun four times daily, with the object of composing the mind to meditation, and of regularising the practices.}

0. These are the adorations to be performed by all aspirants to the $A\therefore{}A\therefore$

1. Let him greet the Sun at dawn, facing East, giving the sign of his grade\editorsnote{There are several possibilities for this if you are not a member of the \Argentium{}. One example, noted in Lon Milo DuQuette's \textit{The Magick of Aleister Crowley}, is to use the four L.V.X. signs at dawn (i.e. the $5^{\circ}=6^{\square}$ signs); the Fire $4^{\circ}=7^{\square}$ sign at noon; the Air $3^{\circ}=8^{\square}$ sign at dusk; and the Water $2^{\circ}=9^{\square}$ sign at midnight. Another possibility is to utilise the signs of the enterer and silence, of $0^{\circ}=0^{\square}$}. And let him say in a loud voice:

\begin{verse}
Hail unto Thee who art Ra in Thy rising, even unto Thee who art Ra in Thy strength, who travellest over the Heavens in Thy bark at the Uprising of the Sun.  \\
Tahuti standeth in His splendour at the prow, and Ra-Hoor abideth at the helm. \\
Hail unto Thee from the Abodes of Night!
\end{verse}

2. Also at Noon, let him greet the Sun, facing South\editorsnote{This depends on your hemisphere on Earth.}, giving the sign of his grade. And let him say in a loud voice:

\begin{verse}
Hail unto Thee who art Ahathoor in Thy triumphing, even unto Thee who art Ahathoor in Thy beauty, who travellest over the heavens in thy bark at the Mid-course of the Sun. \\
Tahuti standeth in His splendour at the prow, and Ra-Hoor abideth at the helm. \\
Hail unto Thee from the Abodes of Morning!
\end{verse}

3. Also, at Sunset, let him greet the Sun, facing West, giving the sign of his grade. And let him say in a loud voice:
\begin{verse}
Hail unto Thee who art Tum in Thy setting, even unto Thee who art Tum in Thy joy, who travellest over the Heavens in Thy bark at the Down-going of the Sun. \\
Tahuti standeth in His splendour at the prow, and Ra-Hoor abideth at the helm. \\
Hail unto Thee from the Abodes of Day!
\end{verse}

4. Lastly, at Midnight, let him greet the Sun, facing North, giving the sign of his grade, and let him say in a loud voice:
\begin{verse}
Hail unto thee who art Khephra in Thy hiding, even unto Thee who art Khephra in Thy silence, who travellest over the heavens in Thy bark at the Midnight Hour of the Sun. \\
Tahuti standeth in His splendour at the prow, and Ra-Hoor abideth at the helm. \\
Hail unto Thee from the Abodes of Evening.
\end{verse}

5. And after each of these invocations thou shalt give the sign of silence, and afterwards thou shalt perform the adoration that is taught thee by thy Superior\editorsnote{If you do not have a superior, there are adorations available online you can find, but you can also compose your own, use the Stele of Revealing, or use the   Liber AL -- see III. 38-39.}\footnotetext{The editor uses portions of David Bowie's \textit{Starman} --- \textsc{Editor.}}. And then do thou compose Thyself to holy meditation.

6. Also it is better if in these adorations thou assume the God-form of Whom thou adorest, as if thou didst unite with Him in the adoration of That which is beyond Him.

7. Thus shalt thou ever be mindful of the Great Work which thou hast undertaken to perform, and thus shalt thou be strengthened to pursue it unto the attainment of the Stone of the Wise, the Summum Bonum, True Wisdom and Perfect Happiness.
