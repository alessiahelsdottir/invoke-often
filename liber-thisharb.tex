\addchap[Liber Thisharb vi\ae{} Memori\ae{}]{Liber \cjRL{ty/s'rb}{} vi\ae{}{} memori\ae{}{}}
\chapnum{CMXIII\footnote{Equinox Vol. 1, No. 7; Magick in Theory and Practise.}}
\textbf{Gives methods for attaining the magical memory, or memory of past lives, and an insight into the function of the Aspirant in this present life.}

000. May be.

00. [It has not been possible to construct this book on a basis of pure Scepticism. This matters less, as the practise leads to Scepticism, and it may be through it.]

0. This book is not intended to lead to the supreme attainment. On the contrary, its results define the separate being of the Exempt Adept from the rest of the Universe, and discover his relation to that Universe.

1. It is of such importance to the Exempt Adept that We cannot overrate it. Let him in no wise adventure the plunge into the Abyss until he have accomplished this to his most perfectest satisfaction.

2. For in the Abyss no effort is anywise possible. The Abyss is passed by virtue of the mass of the Adept and his Karma. Two forces impel him: (1) the attraction of Binah, (2) the impulse of his Karma; and the ease and even the safety of his passage depend on the strength and direction of the latter.

3. Should one rashly dare the passage, and take the irrevocable Oath of the Abyss, he might be lost therein through \AE{}ons of incalculable agony; he might even be thrown back upon Chesed, with the terrible Karma of failure added to his original imperfection.

4. It is even said that in certain circumstances it is possible to fall altogether from the Tree of Life, and to attain the Towers of the Black Brothers. But We hold that this is not possible for any adept who has truly attained his grade, or even for any man who has really sought to help humanity even for a single second,\footnote{Those in possession of Liber CLXXXV. will note that in every grade but one the aspirant is pledged to serve his inferiors in the Order.} and that although his aspiration have been impure through vanity or any similar imperfection.\editorsnote{\textit{Nota bene}, this is an unnuanced view of the Left-Hand Path, which has since been investigated by, among others, the Typhonian Order, Dragon Rouge, and the Temple of Set. This is not intended to disparage Crowley, but a note that those who continued his work found value in going where he dared not.}

5. Let then the Adept who finds the result of these meditations unsatisfactory refuse the Oath of the Abyss, and live so that his Karma gains strength and direction suitable to the task at some future period.

6. Memory is essential to the individual consciousness; otherwise the mind were but a blank sheet on which shadows are cast. But we see that not only does the mind retain impressions, but that it is so constituted that its tendency is to retain some more excellently than others. Thus the great classical scholar, Sir Richard Jebb, was unable to learn even the schoolboy mathematics required for the preliminary examination at Cambridge University, and a special act of the authorities was required in order to admit him.

7. The first method to be described has been detailed in Bhikkhu Ananda Metteya’s \enquote{Training of the Mind}\footnote{The Equinox, Vol 1., No. 5, pp. 28-59, and especially pp. 48-56}. We have little to alter or to add. Its most important result, as regards the Oath of the Abyss, is the freedom from all desire or clinging to anything which it gives. Its second result is to aid the adept in the second method, by supplying him with further data for his investigation.

8. The stimulation of memory useful in both practises is also achieved by simple meditation (Liber Ε), in a certain stage of which old memories arise unbidden. The adept may then practise this, stopping at that stage, and encouraging instead of suppressing the ashes of memory.

9. Zoroaster has said, \enquote{Explore the River of the Soul, whence or in what order you have come; so that although you have become a servant to the body, you may again rise to that Order (the \Argentium{}{}) from which you descended, joining Works (Kamma) to Sacred Reason (the Tao).}

10. The Result of the Second Method is to show the Adept to what end his powers are destined. When he has passed the Abyss and become \textsc{Nemo}, the return of the current causes him \enquote{to appear in the Heaven of Jupiter as a morning star or as an evening star.}\footnote{The formula of the Great Work \enquote{Solve et Coagula,} may be thus interpreted. Solve, the dissolution of the Self in the Infnite; Coagula, the presentation of the Infnite in a concrete form to the outer. Both are necessary to the Task of a Master of the Temple.} should discover what may be the nature of his work. Thus Mohammed was a Brother reflected into Netzach, Buddha a Brother reflected into Hod, or, as some say, Daath. The present manifestation of Frater P. to the outer is in Tiphereth, to the inner in the path of Leo.

11. First Method.  Let the Exempt Adept first train himself to think backwards by external means, as set forth here following.
\begin{enumerate}[label=(\alph*)]
\item Let him learn to write backwards, with either hand.
\item Let him learn to walk backwards.
\item Let him constantly watch, if convenient, cinematograph films, and listen to phonograph records, reversed, and let him so accustom himself to these that they appear natural, and appreciable as a whole.
\item Let him practise speaking backwards; thus, for \enquote{I am He} let him say, \enquote{Eh ma I.}
\item Let him learn to read backwards. In this it is difficult to avoid cheating one’s self, as an expert reader sees a sentence at a glance. Let his disciple read aloud to him backwards, slowly at first, then more quickly.
\item Of his own ingenium let him devise other methods.
\end{enumerate}

12. In this his brain will at first be overwhelmed by a sense of utter confusion; secondly, it will endeavour to evade the difficulty by a trick. The brain will pretend to be working backwards when it is really normal. It is difficult to describe the nature of the trick, but it will be quite obvious to anyone who has done practises (\textit{a}) and (\textit{b}) for a day or two. They become quite easy, and he will think that he is making progress, an illusion which close analysis will dispel.

13. Having begun to train his brain in this manner, and obtained some little success, let the Exempt Adept, seated in his Asana, think first of his present attitude, next of the act of being seated, next of his entering the room, next of his robing, et cetera, exactly as it happened. And let him most strenuously endeavour to think each act as happening backwards. It is not enough to think: \enquote{I am seated here, and before that I was standing, and before that I entered the room,} etc. That series is the trick detected in the preliminary practises.  The series must not run \mbox{\enquote{ghi-def-abc,}} but \mbox{\enquote{ihgfedcba}}: not \enquote{horse a is this}, but \enquote{esroh a si siht.} To obtain this thoroughly well, practise (\textit{c}) is very useful. The brain will be found to struggle constantly to right itself, soon accustoming itself to accept \enquote{esroh} as merely another glyph for \enquote{horse.} This tendency must be constantly combated.

\begin{sloppypar}
14. In the early stages of this practise the endeavour should be to meticulous minuteness of detail in remembering actions; for the brain’s habit of thinking forwards will at first be insuperable. Thinking of large and complex actions, then, will give a series which we may symbolically write \mbox{\enquote{opqrstu-hijklmn-abcdefg.}} If these be split into detail, we shall have \mbox{\enquote{stu-pqr-o--mn-kl-hij--fg-cde-ab}}, which is much nearer to the ideal \mbox{\enquote{utsrqponmlkjihgfedcba.}}
\end{sloppypar}

15. Capacities differ widely, but the Exempt Adept need have no reason to be discouraged if after a month’s continuous labour he find that now and again for a few seconds his brain really works backwards.

16. The Exempt Adept should concentrate his efforts upon obtaining a perfect picture of five minutes backwards rather than upon extending the time covered by his meditation. For this preliminary training of the brain is, the Pons Asinorum of the whole process.

17. This five minutes' exercise being satisfactory, the Exempt Adept may extend the same at his discretion to cover an hour, a day, a week, and so on. Difficulties vanish before him as he advances; the extension from a day to the course of his whole life will not prove so difficult as the perfecting of the five minutes.

18. This practise should be repeated at least four times  daily, and progress is shown firstly by the ever easier running of the brain, secondly by the added memories which arise.

19. It is useful to reflect during this practise, which in time becomes almost mechanical, upon the way in which effects spring from causes. This aids the mind to link its memories, and prepares the adept for the preliminary practise of the Second Method.

20. Having allowed the mind to return for some hundred times to the hour of birth, it should be encouraged to endeavour to penetrate beyond that period. If it be properly trained to run backwards, there will be little difficulty in doing this, although it is one of the distinct steps in the practise.

21. It may be then that the memory will persuade the adept of some previous existence. Where this is possible, let it be checked by an appeal to facts, as follows.

22. It often occurs to men that on visiting a place to which they have never been, it appears familiar. This may arise from a confusion of thought or a slipping of the memory, but it is conceivably a fact.

If, then, the adept \enquote{remember} that he was in a previous life in some city, say Krak\'{o}w\editorsnote{Name modernised.}, which he has in this life never visited, let him describe from memory the appearance of Krak\'{o}w, and of its inhabitants, setting down their names. Let him further enter into details of the city and its customs. And having done this with great minuteness, let him confirm the same by consultation with historians and geographers, or by a personal visit, remembering (both to the credit of his memory and its discredit) that historians, geographers, and himself are alike fallible. But let him not trust his memory to assert its conclusions as fact, and act thereupon, without most adequate confirmation.

23. This process of checking his memory should be practised with the earlier memories of childhood and youth by reference to the memories and records of others, always reflecting upon the fallibility even of such safeguards.

24. All this being perfected, so that the memory reaches back into \ae{}ons incalculably distant, let the Exempt Adept meditate upon the fruitlessness of all those years, and upon the fruit thereof, severing that which is transitory and worthless from that which is eternal. And it may be that he being but an Exempt Adept may hold all to be savourless and full of sorrow.

25. This being so, without reluctance will he swear the Oath of the Abyss.

26. Second Method. Let the Exempt Adept, fortified by the practise of the First Method, enter the preliminary practise of the Second Method.

27. Second Method. Preliminary Practises. Let him, seated in his Asana, consider any event, and trace it to its immediate causes. And let this be done very fully and minutely. Here, for example, is a body erect and motionless. Let the adept consider the many forces which maintain it; firstly, the attraction of the earth, of the sun, of the planets, of the farthest stars, nay, of every mote of dust in the room, one of which (could it be annihilated) would cause that body to move, although so imperceptibly. Also, the resistance of the floor, the pressure of the air, and all other external conditions. Secondly, the internal forces which sustain it, the vast and complex machinery of the skeleton, the muscles, the blood, the lymph, the marrow, all that makes up a man. Thirdly, the moral and intellectual forces involved, the mind, the will, the consciousness. Let him continue this with unremitting ardour, searching Nature, leaving nothing out.

28. Next let him take one of the immediate causes of his position, and trace out its equilibrium. For example, the will. What determines the will to aid in holding the body erect and motionless?

29. This being determined, let him choose one of the forces which determined his will, and trace out that in similar fashion; and let this process be continued for many days until the interdependence of all things is a truth assimilated in his inmost being.

30. This being accomplished, let him trace his own history with special reference to the causes of each event. And in this practise he may neglect to some extent the universal forces which at all times act on all, as for example the attraction of masses, and let him concentrate his attention upon the principal and determining or effective causes.

For instance, he is seated, perhaps, in a country place in Spain. Why? Because Spain is warm and suitable for meditation, and because cities are noisy and crowded. Why is Spain warm? and why does he wish to meditate? Why choose warm Spain rather than warm India? To the last question: Because Spain is nearer to his home. Then why is his home near Spain? Because his parents were Germans. And why did they go to Germany? And so during the whole meditation.

31. On another day, let him begin with a question of another kind, and every day devise new questions, not only concerning his present situation, but also abstract questions. Thus let him connect the prevalence of water upon the surface of the globe with its necessity to such life as we know, with the specific gravity and other physical properties of water, and let him perceive ultimately through all this the necessity and concord of things, not concord as the schoolmen of old believed, making all things for man’s benefit or convenience, but the essential mechanical concord whose final law is \textit{inertia}. And in these meditations let him avoid as if it were the plague any speculation sentimental or fantastic.

32. Second Method. The Practise Proper. Having then perfected in his mind these conceptions, let him apply them to his own career, forging the links of memory into the chain of necessity. And let this be his final question: To what purpose am I fitted? Of what service can my being prove to the Brothers of the \Argentium{}{} if I cross the Abyss, and am admitted to the City of the Pyramids?

33. Now that he may clearly understand the nature of this question, and the method of so

lution, let him study the reasoning of the anatomist who reconstructs an animal from a single bone. To take a simple example.

34. Suppose, having lived all my life among savages, a ship is cast upon the shore and wrecked. Undamaged among the cargo is a \enquote{Victoria.} What is its use? The wheels speak of roads, their slimness of smooth roads, the brake of hilly roads. The shafts show that it was meant to be drawn by an animal, their height and length suggest an animal of the size of a horse. That the carriage is open suggests a climate tolerable at any rate for part of the year. The height of the box suggests crowded streets, or the spirited character of the animal employed to draw it. The cushions indicate its use to convey men rather than merchandise; its hood that rain sometimes falls, or that the sun is at times powerful. The springs would imply considerable skill in metals; the varnish much attainment in that craft.

35. Similarly, let the adept consider of his own case. Now that he is on the point of plunging into the Abyss, a giant Why? confronts him with uplifted club.

36. There is no minutest atom of his composition which can be withdrawn Without making him some other than he is, no useless moment in his past. Then what is his future? The \enquote{Victoria} is not a wagon; it is not intended for carting hay. It is not a sulky; it is useless in trotting races.

37. So the adept has military genius, or much knowledge of Greek: how do these attainments help his purpose, or the purpose of the Brothers? He was put to death by Calvin, or stoned by Hezekiah; as a snake he was killed by a villager, or as an elephant slain in battle under Hamilcar. How do such memories help him ? Until he have thoroughly mastered the reason for every incident in his past, and found a purpose for every item of his present equipment\footnote{A Brother known to me was repeatedly baffled in this meditation. But one day being thrown with his horse over a sheer cliff of forty feet, and escaping without a scratch or a bruise, he was reminded of his many narrow escapes from death. These proved to be the last factors in his problem, which, thus completed, solved itself in a moment. O.M. \textsc{[Equinox]}}, he cannot truly answer even those Three Questions that were first put to him, even the Three Questions of the Ritual of the Pyramid he is not ready to swear the Oath of the Abyss.

38. But being thus enlightened, let him swear the Oath of the Abyss; yea, let him swear the Oath of the Abyss.

