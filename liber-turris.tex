\addchap{Liber Turris vel Domus Dei}\index{Grade Studies \& Work!Practicus (\grade{3}{8})}\index{Grade Studies \& Work!Philosophus (\grade{4}{7})}
\chapnum{XVI\footnote{The title may be read as \enquote{Book of the Tower or House of God}. The key of the Tarot numbered XVI is the Lightning Struck Tower. From the Equinox Vol. 1, No. 6.}}

\textbf{An instruction for attainment by the direct destruction of thoughts as they arise in the mind.}

0. This practice is very difficult. The student cannot hope for much success unless he have thoroughly mastered Asana, and obtained much definite success in the meditation-practices of Liber E and Liber HHH.

On the other hand, any success in this practice is of an exceedingly high character, and the student is less liable to illusion and self-deception in this than in almost any other that We make known.\footnote{The meditation practice in Liber E consisted in the restraint of the mind to a single predetermined imagined object exterior to the student, simple or complex, at rest or in motion; those of Liber HHH in causing the mind to pass through a predetermined series of states; the Raja-Yoga of the Hindus is mainly an extension of the methods of Liber E to interior objects; the Mahasatipatthana of the Buddhists is primarily an observation and analysis of bodily movements. While the present practice differs radically from all of these, it is of the greatest advantage to be acquainted practically with each of them, with regard firstly to their incidental difficulties, and secondly to their ascertained results in respect of psychology. \textsc{Ed [Equinox]}.}

1. First Point. The student should first discover for himself the apparent position of the point in his brain where thoughts arise, if there be such a point.

If not, he should see the position of the point where thoughts are judged.

2. Second Point. He must also develop in himself a Will of Destruction, even a Will of Annihilation. It may be that this shall be discovered at an immeasurable distance from his physical body. Nevertheless, this must he reach, with this must he identify himself even to the loss of himself.

3. Third Point. Let this Will then watch vigilantly the point where thoughts arise, or the point where they are judged, and let every thought be annihilated as it is perceived or judged.\footnote{This is also the \enquote{Opening of the Eye of Shiva.}\footnotemark \textsc{Ed [Equinox]}.}
\footnotetext{\enquote{which is also the Eye of Set, because it sucks into itself all the Light that Horus has projected.} \textendash{} Kenneth Grant, The Magical Revival. \textemdash{} \textsc{Editor.}}

4. Fourth Point. Next, let every thought be inhibited in its inception.

5. Fifth Point. Next, let even the causes or tendencies that if unchecked ultimate in thoughts be discovered and annihilated.

6. Sixth and Last Point. Let the true Cause of All\footnote{Mayan, the Magician, or Mara. Also the Dweller on the Threshold in a very exalted sense. \textsc{Ed [Equinox]}.} be unmasked and annihilated.

7. This is that which was spoken by wise men of old time concerning the destruction of the World by fire; yea, the destruction of the world by fire.

8. [This and the following verses are of modern origin]

Let the Student remember that each Point represents a definite achievement of great difficulty.

9. Let him not then attempt the second until he be well satisfied of his mastery over the first.

10. This practice is also that which was spoken by Fra P. in a parable as followeth: \begin{tightverse}
\PParNum{11.} Foul is the robber stronghold, filled with hate; \\
Thief strangling thief, and mate at war with mate, \\
Fronting wild raiders, all forlorn to Fate!

There is nor health nor happiness therein. \\
Manhood is cowardice, and virtue sin. \\
Intolerable blackness hems it in.

Not hell's heart hath so noxious a shade; \\
Yet harmless and unharmed, and undismayed, \\
Pines in her prison an unsullied maid.

Penned by the master mage to his desire, \\
She baffles his seductions and his ire, \\
Praying God's all-annihilating fire.

The Lord of Hosts gave ear unto her song: \\
The Lord of Hosts waxed wrathful at her wrong. \\
He loosed the hound of heaven from its thong.

Violent and vivid smote the levin ash. \\
Once the tower rocked and cracked beneath its lash, \\
Caught inextinguishable fire; was ash.

But that same fire that quelled the robber strife, \\
And struck each being out of lust and life, \\
Left the mild maiden a rejoicing wife.
\end{tightverse}

12. And this:
\begin{tightverse}
\PParNum{13.} There is a well before the Great White Throne \\
That is choked up with rubbish from the ages; \\
Rubble and clay and sediment and stone, \\
Delight of lizards and despair of sages.

Only the lightning from His hand that sits, \\
And shall sit when the usurping tyrant falls, \\
Can purge that Wilderness of wills and wits, \\
Let spring that fountain in eternal halls. \\
\end{tightverse}

14. And this:
\begin{tightverse}
\PParNum{15.} Sulphur, Salt, and Mercury: \\
 Which is master of the three?

 Salt is Lady of the Sea; \\
 Lord of Air is Mercury.

 Now by God's grace here is salt \\
 Fixed beneath the violet vault.

 Now by God's love purge it through \\
 With our right Hermetic dew.

 Now by God wherein we trust \\
 Be our sophic salt combust.

 Then at last the Eye shall see \\
 Three in One and One in Three,

 Sulphur, Salt, and Mercury, \\
 Crowned by Heavenly Alchemy!

 To the One who sent the Seven \\
 Glory in the Highest Heaven!

 To the Seven who are the Ten \\
 Glory on the Earth, Amen!
\end{tightverse}

16. And of the difficulties of this practice and of the Results that reward it, let these things be discovered by the right Ingenium of the Practicus.
