\addchap{The Star Ruby}
\chapnum{XXV\footnote{The Book of Lies; Magick in Theory and Practice.}}

\epigraph{25 is the square of 5, and the Pentagram has the red colour of Geburah. The chapter is a new and more elaborate version of the Banishing Ritual of the Pentagram.}{The Book of Lies}


Facing East, in the centre, draw deep deep deep thy breath closing thy mouth with thy right forefinger prest against thy lower lip. Then dashing down the hand with a great sweep back and out, expelling forcibly thy breath, cry


\begin{quoting}[indentfirst=false]
\textsc{Απο παντος κακοδαιμονος}\footnote{\enquote{Apo Pantos Kakodaimonos}: \enquote{Away every evil spirit}}
\end{quoting}

With the same forefinger touch thy forehead, and say
\begin{quoting}[indentfirst=false]
\textsc{Σοι}\footnote{\enquote{Soi}}
\end{quoting}

thy member, and say
\begin{quoting}[indentfirst=false]
\textsc{Ω φαλλη}\footnote{\enquote{O Phall\=e}}
\end{quoting}

thy right shoulder, and say
\begin{quoting}[indentfirst=false]
\textsc{Ισχυροσ}\footnote{\enquote{Ischuros}}
\end{quoting}

thy left shoulder, and say
\begin{quoting}[indentfirst=false]
\textsc{Ευχαριστοσ}\footnote{\enquote{Eucharistos}}
\end{quoting}

then clasp thine hands, locking the fingers, and cry
\begin{quoting}[indentfirst=false]
\textsc{Ιαω}\footnote{\enquote{IAO}. See Magick, Chapter V.}
\end{quoting}

Advance to the East. Imagine strongly a Pentagram, aright, in thy forehead. Drawing the hands to the eyes, fling it forth, making the sign of Horus and roar
\begin{quoting}[indentfirst=false]
\textsc{Χαοσ}\footnote{\enquote{Chaos}; in another text \enquote{roar Therion} which may be used instead.}
\end{quoting}

Retire thine hand in the sign of Hoor pa kraat\footnote{Later \enquote{Hoor-paar-Kraat}}. Go round to the North and repeat; but scream \begin{quoting}[indentfirst=false]\textsc{\GreekBabalon}\footnote{\enquote{Babalon}; in another text \enquote{say Nuit} which may be used instead.} \end{quoting}

Go round to the West and repeat; but say \begin{quoting}[indentfirst=false]\textsc{Ερωσ}\footnote{\enquote{Eros}; in another text \enquote{whisper Babalon} which may be used instead.} \end{quoting}

Go round to the South and repeat; but bellow \begin{quoting}[indentfirst=false]\textsc{Ψυχη}\footnote{\enquote{Psych\=e}; in another text \enquote{bellow Hadit} which may be used instead.} \end{quoting}

Completing the circle widdershins, retire to the centre, and raise thy voice in the Paian, with these words, with the signs of N.O.X. \begin{quoting}[indentfirst=false]\textsc{Ιω παν}\footnote{\enquote{Io Pan!}: A salute to Pan}\end{quoting}

Extend the arms in the form of a Tau, and say low but clear:

\begin{quoting}[indentfirst=false]
\textsc{Προ μου ιυγγεσ} \\
\textsc{Οπισω μου τελεταρχαι} \\
\textsc{Επι δεξια συνοχεσ} \\
\textsc{Επαριστερα δαιμονεσ} \\
\textsc{Φλεγει γαρ περι μου ο αστηρ των πεντε} \\
\textsc{Και εν τηι στηληι ο αστηρ των εξ εστηκε}\footnotemark
\end{quoting}
\footnotetext{
\enquote{Pro Mou Iugges \\
Opiso Mou Teletarchai \\
Epi Dexia Synoches \\
Eparistera Daimones \\
Phlegei Gar Peri Mou O Ast\=er Ton Pente \\
Kai En T\=ei St\=el\=ei O Ast\=er Ton Ex Est\=eke\footnotemark}}
\footnotetext{\enquote{Before me the Iynges, behind me the Teletarches, on my right hand the Synoches, on my left the daemons for about me flame the Star of Five, and in the pillar stands the Star of Six.}}

Repeat the Cross Qabalistic, as above, and end as thou didst begin.
