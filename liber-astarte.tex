\addchap{Liber Astart\'{e} vel Berylli}

\chapnum{CLXXV\footnote{Equinox Vol. 1, No. 10; Magick in Theory and Practice.}}



0. \textbf{This is the Book of Uniting Himself to a particular Deity by devotion.}

1. \textit{Considerations before the Threshold.} First concerning the choice of a particular Deity. This matter is of no import, sobeit that thou choose one suited to thine own highest nature. Howsoever, this method is not so suitable for gods austere as Saturn, or intellectual as Thoth, but for such deities as in themselves partake in anywise of love it is a perfect mode.

2. \textit{Concerning the prime method of this Magick Art.} Let the devotee consider well that although Christ and Osiris be one, yet the former is to be worshipped with Christian, and the latter with Egyptian rites. And this although the rites themselves are ceremonially equivalent. There should, however, be one symbol declaring the transcending of such limitations; and with regard to the Deity also, there should be some one affirmation of his identity both with all other similar gods of other nations, and with the Supreme of whom all are but partial reflections.

3. \textit{Concerning the chief place of devotion.} This is the Heart of the devotee, and should be symbolically represented by that room or spot which he loves best. And the dearest spot therein shall be the shrine of his temple. It is most convenient if this shrine and altar should be sequestered in woods, or in a private grove, or garden. But let it be protected from the profane.

4. \textit{Concerning the Image of the Deity.} Let there be an image of the Deity; first because in meditation there is mindfulness induced thereby; and second because a certain power enters and inhabits it by virtue of the ceremonies; or so it is said, and We deny it not. Let this image be the most beautiful and perfect which the devotee is able to procure; or if he be able to paint or to carve the same, it is all the better. As for Deities with whose nature no Image is compatible, let them be worshipped in an empty shrine. Such are Brahma, and Allah. Also some post-captivity conceptions of Jehovah.

5. \textit{Further concerning the shrine.} Let this shrine be furnished appropriately as to its ornaments, according to Liber 777. With ivy and pine-cones, that is to say, for Bacchus, and let lay before him both grapes and wine. So also for Ceres let there be corn, and cakes; or for Diana moon-wort and pale herbs, and pure water. Further, it is well to support the shrine with talismans of the planets, signs and elements appropriate. But these should be made according to the right Ingenium of the Philosophus by the light of the book 777 during the course of his Devotion. It is also well, nevertheless, if a magick circle with the right signs and names be made beforehand.

6. \textit{Concerning the Ceremonies.} Let the Philosophus prepare a powerful Invocation of the particular Deity according to his Ingenium. But let it consist of these several parts:
\begin{description}
\item First, an Imprecation, as of a slave unto his Lord.
\item Second, an Oath, as of a vassal to his Liege.
\item Third, a Memorial, as of a child to his Parent.
\item Fourth, an Orison, as of a Priest unto his God.
\item Fifth, a Colloquy, as of a Brother with his Brother.
\item Sixth, a Conjuration, as to a Friend with his Friend.
\item Seventh, a Madrigal, as of a Lover to his Mistress.
\end{description}

And mark well that the first should be of awe, the second of fealty, the third of dependence, the fourth of adoration, the fifth of confidence, the sixth of comradeship, the seventh of passion.

7. \textit{Further concerning the ceremonies.} Let then this Invocation be the principal part of an ordered ceremony. And in this ceremony let the Philosophus in no wise neglect the service of a menial. Let him sweep and garnish the place, sprinkling it with water or with wine as is appropriate to the particular Deity, and consecrating it with oil, and with such ritual as may seem him best. And let all be done with intensity and minuteness.

8. \textit{Concerning the period of devotion, and the hours thereof.} Let a fixed period be set for the worship; and it is said that the least time is nine days by seven, and the greatest seven years by nine. And concerning the hours, let the Ceremony be performed every day thrice, or at least once, and let the sleep of the Philosophus be broken for some purpose of devotion at least once in every night.

Now to some it may seem best to appoint fixed hours for the ceremony, to others it may seem that the ceremony should be performed as the spirit moves them so to do: for this there is no rule.

9. \textit{Concerning the Robes and Instruments.} The Wand and Cup are to be chosen for this Art; never the Sword or Dagger, never the Pantacle, unless that Pantacle chance to be of a nature harmonious. But even so it is best to keep the Wand and Cup; and if one must choose, the Cup.

For the Robes, that of a Philosophus, or that of an Adept Within is most suitable; or, the robe best fitted for the service of the particular Deity, as a bassara for Bacchus, a white robe for Vesta. So also, for Vesta, one might use for instrument the Lamp; or the sickle, for Chronos.

10. \textit{Concerning the Incense and Libations.} The incense should follow the nature of the particular Deity; as, mastic for Mercury, dittany for Persephone. Also the libations, as, a decoction of nightshade for Melancholia, or of Indian hemp for Uranus.

11. \textit{Concerning the harmony of the ceremonies.} Let all these things be rightly considered, and at length, in language of the utmost beauty at the command of the Philosophus, accompanied, if he has skill, by music, and interwoven, if the particular Deity be jocund, with dancing. And all being carefully prepared and rehearsed, let it be practiced daily until it be wholly rhythmical with his aspiration, and as it were, a part of his being.

12. \textit{Concerning the variety of the ceremonies.} Now, seeing that every man differeth essentially from every other man, albeit in essence he is identical, let also these ceremonies assert their identity by their diversity. For this reason do We leave much herein to the right Ingenium of the Philosophus.

13. \textit{Concerning the life of the devotee.} First, let his way of life be such as is pleasing to the particular Deity. Thus to invoke Neptune, let him go a-fishing; but if Hades, let him not approach the water that is hateful to Him.

14. \textit{Further, concerning the life of the devotee.} Let him cut away from his life any act, word, or thought, that is hateful to the particular Deity; as, unchastity in the case of Artemis, evasions in the case of Ares. Besides this, he should avoid all harshness or unkindness of any kind in thought, word, or deed, seeing that above the particular Deity is One in whom all is One. Yet also he may deliberately practice cruelties, where the particular Deity manifests His Love in that manner, as in the case of Kali, and of Pan. And therefore, before the beginning of his period of devotion, let him practice according to the rules of Liber Jugorum.

15. \textit{Further concerning the life of the devotee.} Now, as many are fully occupied with their affairs, let it be known that this method is adaptable to the necessities of all.

And We bear witness that this which followeth is the Crux and Quin\-tess\-ence of the whole Method.

First, if he have no Image, let him take anything soever, and consecrate it as an Image of his God. Likewise with his robes and instruments, his suffumigations and libations: for his Robe hath he not a nightdress; for his instrument a walking stick; for his suffumigation a burning match; for his libation a glass of water?

But let him consecrate each thing that he useth to the service of that particular Deity, and not profane the same to any other use.

16. \textit{Continuation.} Next, concerning his time, if it be short. Let him labour mentally upon his Invocation, concentrating it, and let him perform this Invocation in his heart whenever he hath the leisure. And let him seize eagerly upon every opportunity for this.

17. \textit{Continuation.} Third, even if he have leisure and preparation, let him seek ever to bring inward the symbols, so that even in his well-ordered shrine the whole ceremony revolve inwardly in his heart, that is to say in the temple of his body, of which the outer temple is but an image.

For in the brain is the shrine, and there is no Image therein; and the breath of man is the incense and the libation.

18. \textit{Continuation.} Further concerning occupation. Let the devotee transmute within the alembic of his heart every thought, or word, or act into the spiritual gold of his devotion.

As thus: eating. Let him say: \enquote{I eat this food in gratitude to my Deity that hath sent it to me, in order to gain strength for my devotion to Him.}

Or: sleeping. Let him say: \enquote{I lie down to sleep, giving thanks for this blessing from my Deity, in order that I may be refreshed for new devotion to Him.}

Or: reading. Let him say: \enquote{I read this book that I may study the nature of my Deity, that further knowledge of Him may inspire me with deeper devotion to Him.}

Or: working. Let him say: \enquote{I drive my spade into the earth that fresh flowers (fruit, or what not) may spring up to His glory, and that I, purified by toil, may give better devotion to Him.}

Or: whatever it may be that he is doing, let him reason it out in his own mind, drawing it through circumstance and circumstance to that one end and conclusion of the matter. And let him not perform the act until he hath done this.

As it is written: Liber VII.\editorsnote{Liber Liberi vel Lapdis Lazuli.} cap. v. \textemdash{}
\begin{quoting}[indentfirst=false]
22. \enquote{Every breath, every word, every thought is an act of love with thee.} \\
23. \enquote{The beat of my heart is the pendulum of love.} \\
24. \enquote{The songs of me are the soft sighs:} \\
25. \enquote{The thoughts of me are very rapture:} \\
26. \enquote{And my deeds are the myriads of Thy Children, the stars and the atoms.}
\end{quoting}

And Remember Well, that if thou wert in truth a lover, all this wouldst thou do of thine own nature without the slightest flaw or failure in the minutest part thereof.

19. \textit{Concerning the Lections.} Let the Philosophus read solely in his copies of the holy books of Thelema, during the whole period of his devotion. But if he weary, then let him read books which have no part whatever in love, as for recreation.

But let him copy out each verse of Thelema which bears upon this matter, and ponder them, and comment thereupon. For therein is a wisdom and a magic too deep to utter in any other wise.

20. \textit{Concerning the Meditations.} Herein is the most potent method of attaining unto the End, for him who is thoroughly prepared, being purified by the practice of the Transmutation of deed into devotion, and consecrated by the right performance of the holy ceremonies. Yet herein is danger, for that the Mind is fluid as quicksilver, and bordereth upon the Abyss, and is beset by many sirens and devils that seduce and attack it to destroy it. Therefore let the devotee beware, and precise accurately his meditations, even as a man should build a canal from sea to sea.

21. \textit{Continuation.} Let then the Philosophus meditate upon all love that hath ever stirred him. There is the love of David and of Jonathan, and the love of Abraham and Isaac, and the love of Lear and Cordelia, and the love of Damon and Pythias, and the love of Sappho and Atthis, and the love of Romeo and Juliet, and the love of Dante and Beatrice, and the love of Paolo and Francesca, and the love of C\ae{}sar and Lucrezia Borgia, and the love of Aucassin and Nicolette, and the love of Daphnis and Chloe, and the love of Cornelia and Caius Gracchus, and the love of Bacchus and Ariadne, and the love of Cupid and Psyche, and the love of Endymion and Artemis, and the love of Demeter and Persephone, and the love of Venus and Adonis, and the love of Lakshmi and Vishnu, and the love of Siva and Bhavani, and the love of Buddha and Ananda, and the love of Jesus and John, and many more.

Also there is the love of many saints for their particular deity, as of St. Francis of Assisi for Christ, of Sri Sabhapaty Swami for Maheswara, of Abdullah Haji Shirazi for Allah, of St Ignatius Loyola for Mary, and many more.

Now do thou take one such story every night, and enact it in thy mind, grasping each identity with infinite care and zest, and do thou figure thyself as one of the lovers and thy Deity as the other. Thus do thou pass through all adventures of love, not omitting one; and to each do thou conclude: How pale a reflection is this of my love for this Deity!

Yet from each shalt thou draw some knowledge of love, some intimacy with love, that shall aid thee to perfect thy love. Thus learn the humility of love from one, its obedience from another, its intensity from a third, its purity from a fourth, its peace from yet a fifth.

So then thy love being made perfect, it shall be worthy of that perfect love of His.

22. \textit{Further concerning meditation.} Moreover let the Philosophus imagine to himself that he hath indeed succeeded in his devotion, and that his Lord hath appeared to him, and that they converse as may be fitting.

23. \textit{Concerning the Mysterious Triangle.} Now then as three cords separately may be broken by a child, while those same cords duly twisted may bind a giant, let the Philosophus learn to entwine these three methods of Magic into a Spell.

To this end let him understand that as they are One, because the end is one, so are they One because the method is One, even the method of turning the mind toward the particular Deity by love in every act.

And lest thy twine slip, here is a little cord that wrappeth tightly round and round all, even the Mantram or Continuous Prayer.

24. \textit{Concerning the Mantram or Continuous Prayer.} Let the Philosophus weave the Name of the Particular Deity into a sentence short and rhythmical, as, for Artemis: \textgreek{επελθον}, \textgreek{επελθον}, \textgreek{Αρτεμις}; or, for Shiva: Namo Shivaya namaha Aum; or, for Mary: Ave Maria; or for Pan: \textgreek{χαιπε Σωτηρ κοσμον Ιω Παν Ιω Παν}; or, for Allah: Hua Allahu alazi lailaha illa Hua.

Let him repeat this day and night without cessation mechanically in his brain, which is thus made ready for the Advent of that Lord, and armed against all other.

25. \textit{Concerning the Active and the Passive.} Let the Philosophus change from the active love of his particular Deity to a state of passive waiting, even almost a repulsion, the repulsion not of distaste, but of sublime modesty.

As it is written, Liber LXV.\editorsnote{Liber Cordis Cincti Serpente sub figura \cjRL{'dny}} ii. 59:
\begin{quoting}[indentfirst=false]
59. I have called unto Thee, and I have journeyed unto Thee, and it availed me not. \\
60. I waited patiently, and Thou wast with me from the beginning.
\end{quoting}

Then let him change back to the Active, until a veritable rhythm is established between the states, as it were the swinging of a Pendulum. But let him reflect that a vast intelligence is required for this; for he must stand as it were almost without himself to watch those phases of himself, And to do this is a high Art, and pertaineth not altogether to the grade of Philosophus. Neither is it of itself helpful, but rather the reverse, in this especial practice.

26. \textit{Concerning silence.} Now there may come a time in the course of this practice when the outward symbols of devotion cease, when the soul is as it were dumb in the presence of its God. Mark that this is not a cessation, but a transmutation of the barren seed of prayer into the green shoot of yearning. This yearning is spontaneous, and it shall be left to grow, whether it be sweet or bitter. For often times it is as the torment of hell in which the soul burns and writhes unceasingly. Yet it ends, and at its end continue openly thy Method.

27. \textit{Concerning Dryness.} Another state wherein at times the soul may fall is this dark night. And this is indeed purifying in such depths that the soul cannot fathom it. It is less like pain than like death. But it is the necessary death that comes before the rising of a body glorified.

This state must be endured with fortitude; and no means of alleviating it may be employed. It may be broken up by the breaking up of the whole Method, and a return to the world without. This cowardice not only destroys the value of all that has gone before, but destroys the value of the Oath of Fealty that thou hast sworn, and makes thy Will a mockery to men and gods.

28. \textit{Concerning the Deceptions of the Devil.} Note well that in this state of dryness a thousand seductions will lure thee away; also a thousand means of breaking thine oath in spirit without breaking it in letter. Against this thou mayst repeat the words of thine oath aloud again and again until the temptation be overcome.

Also the devil will represent to thee that it were much better for this operation that thou do thus and thus, and seek to affright thee by fears for thy health or thy reason.

Or he may send against thee visions worse than madness.

Against all this there is but one remedy, the Discipline of thine Oath. So then thou shalt go through ceremonies meaningless and hideous to thee, and blaspheme shalt thou against thy Deity and curse Him. And this mattereth little, for it is not thou, so be that thou adhere to the Letter of thine Obligation. For thy Spiritual Sight is closed, and to trust it is to be led unto the precipice, and hurled therefrom.

29. \textit{Further of this matter.} Now also subtler than all these terrors are the Illusions of Success. But one instant's self-satisfaction or Expansion of thy Spirit, especially in this state of dryness, and thou art lost. For thou mayst attain the False Union with the Demon himself. Beware also of even the pride which rises from having resisted the temptations.

But so many and so subtle are the wiles of Choronzon that the whole world could not contain their enumeration.

The answer to one and all is the persistence in the literal fulfilment of the routine. Beware, then, last, of that devil who shall whisper in thine ear that the letter killeth, but the spirit giveth life, and answer: Except a corn of wheat fall into the ground and die, it abideth alone, but if it die, it bringeth forth much fruit.

Yet shalt thou also beware of disputation with the devil, and pride in the cleverness of thine answers to him. Therefore, if thou hast not lost the power of silence, let it be first and last employed against him.

30. \textit{Concerning the Enflaming of the Heart.} Now learn that thy methods are dry, one and all. Intellectual exercises, moral exercises, they are not Love. Yet as a man, rubbing two dry sticks together for long, suddenly found a spark, so also from time to time will true love leap unasked into thy mediation. Yet this shall die and be reborn again and again. It may be that thou hast no tinder near.

In the end shall come suddenly a great flame and a devouring, and burn thee utterly.

Now of these sparks, and of these splutterings of flame, and of these beginnings of the Infinite Fire, thou shalt thus be aware. For the sparks thy heart shall leap up, and thy ceremony or meditation or toil shall seem of a sudden to go of its own will; and for the little flames this shall be increased in volume and intensity; and for the beginnings of the Infinite Fire thy ceremony shall be caught up unto ravishing song, and thy meditation shall be ecstasy, and thy toil shall be a delight exceeding all pleasure thou hast ever known.

And of the Great Flame that answereth thee it may not be spoken; for therein is the End of this Magick Art of Devotion.

31. \textit{Considerations with regard to the use of symbols.} It is to be noted that persons of powerful imagination, will, and intelligence have no need of these material symbols. There have been certain saints who are capable of love for an idea as such without it being otherwise than degraded by \enquote{idolising} it, to use this word in its true sense. Thus one may be impassioned of beauty, without even the need of so small a concretion of it as \enquote{The beauty of Apollo}, the \enquote{beauty of roses}, the \enquote{beauty of Attis}. Such persons are rare; it may be doubted whether Plato himself attained to any vision of absolute beauty without attaching to it material objects in the first place. A second class is able to contemplate ideals through this veil; a third class need a double veil, and cannot think of the beauty of a rose without a rose before them. For such is this Method of most use; yet let them know that there is this danger therein, that they may mistake the gross body of the symbol for the idea made concrete thereby.

32. \textit{Considerations of further danger to those not purged of material thought.} Let it be remembered that in the nature of the love itself is danger. The lust of the satyr for the nymph is indeed of the same nature as the affinity of Quicklime for water on the one hand, and of love of Ab for Ama on the other; so also is the triad Osiris, Isis, Horus like that of a horse, mare, foal, and of red, blue, purple. And this is the foundation of Correspondences.

But it were false to say \enquote{Horus is a foal} or \enquote{Horus is purple}. One may say: \enquote{Horus resembles a foal in this respect, that he is the offspring of two complementary beings}.

33. \textit{Further of this matter.} So also many have said truly that all is one, and falsely that since earth is That One, and ocean is That One, therefore earth is ocean. Unto Him good is illusion, and evil is illusion; therefore good is evil. By this fallacy of logic are many men destroyed.

Moreover, there are those who take the image for the God; as who should say, my heart is in Tiphereth, and an Adeptus is in Tiphereth; I am therefore an Adept.

And in this practice the worst danger is this, that the love which is its weapon should fail in one of two ways.

First, if the love lack any quality of love, so long is it not ideal love. For it is written of the Perfected One: \enquote{There is no member of my body which is not the member of some god.} Therefore let not the Philosophus despise any form of love, but harmonise all. As it is written in Liber LXV, 32. \enquote{So therefore Perfection abideth not in the Pinnacles or in the Foundation, but in the harmony of One with all.}

Second, if any part of this love exceed, there is disease therein. As, in the love of Othello for Desdemona, love's jealousy overcame love's tenderness, so may it be in this love of a particular Deity. And this is more likely, since in this divine love no element may be omitted.

It is by virtue of this completeness that no human love may in any way attain to more than to forthshadow a little part thereof.

34. \textit{Concerning Mortifications.} These are not necessary to this method. On the contrary, they may destroy the concentration, as counter-irritants to, and so alleviations of, the supreme mortification which is the Absence of the Deity invoked.

Yet as in mortal love arises a distaste for food, or a pleasure in things naturally painful, this perversion should be endured and allowed to take its course. Yet not to the interference with natural bodily health, whereby the instrument of the soul might be impaired.

And concerning sacrifices for love's sake, they are natural to this Method, and right.

But concerning voluntary privations and tortures, without use save as against the devotee, they are generally not natural to healthy natures, and wrong. For they are selfish. To scourge one's self serves not one's master; yet to deny one's self bread that one's child may have cake is the act of a true mother.

35. \textit{Further concerning Mortifications.} If thy body, on which thou ridest, be so disobedient a beast that by no means will he travel in the desired direction, or if thy mind be baulkish and eloquent as Balaam's fabled Ass, then let the practice be abandoned. Let the shrine be covered in sackcloth, and do thou put on habits of lamentation, and abide alone. And do thou return most austerely to the practice of Liber Jugorum, testing thyself by a standard higher than that hitherto accomplished, and punishing effractions with a heavier goad. Nor do thou return to thy devotion until that body and mind are tamed and trained to all manner of peaceable going.


36. \textit{Concerning minor methods adjuvant in the ceremonies.}

\textit{I. Rising on the planes.}


By this method mayst thou assist the imagination at the time of concluding thine Invocation. Act as taught in Liber O, by the light of Liber 777.


37. \textit{Concerning minor methods adjuvant in the ceremonies.}

\textit{II. Talismanic Magic.}


Having made by thine Ingenium a talisman or pantacle to represent the particular Deity, and consecrated it with infinite love and care, do thou burn it ceremonially before the shrine, as if thereby giving up the shadow for the substance. But it is useless to do this unless thou do really in thine heart value the talisman beyond all else that thou hast.


38. \textit{Concerning minor methods adjuvant in the ceremonies.}

\textit{III. Rehearsal.}


It may assist if the traditional history of the particular Deity be rehearsed before him; perhaps this is best done in dramatic form. This method is the main one recommended in the \enquote{Exercitios Espirituales} of St Ignatius, whose work may be taken as a model. Let the Philosophus work out the legend of his own particular Deity, and apportioning days to events, live that life in imagination, exercising the five senses in turn, as occasion arises.


39. \textit{Concerning minor methods adjuvant in the ceremonies.}

\textit{IV. Duresse.}


This method consists in cursing a deity recalcitrant; as, threatening ceremonially \enquote{to burn the blood of Osiris, and to grind down his bones to power.} This method is altogether contrary to the spirit of love unless the particular Deity be himself savage and relentless; as Jehovah or Kali. In such a case the desire to perform constraint and cursing may be the sign of the assimilation of the spirit of the devotee with that of his God, and so an advance to the Union with Him.

40. \textit{Concerning the value of this particular form of Union or Samadhi.} All Samadhi is defined as the ecstatic union of a subject and object in consciousness, with the result that a third thing arises which partakes in no way of the nature of the two.

It would seem at first sight that it is of no importance whatever to choose an object of meditation. For example, the Samadhi called Atmadarshana might arise from simple concentration of the thought on an imagined triangle, or on the heart.

But as the union of two bodies in chemistry may be endothermic or exothermic, the combination of Oxygen with Nitrogen is gentle, while that of Oxygen with Hydrogen is explosive; and as it is found that the most heat is disengaged as a rule by the union of bodies most opposite in character, and that the compound resulting from such is most stable, so it seems reasonable to suggest that the most important and enduring Samadhi results from the contemplation of the Object most opposite to the devotee. [On other planes, it has been suggested that the most opposed types make the best marriages and produce the healthiest children. The greatest pictures and operas are those in which violent extremes are blended, and so generally in every field of activity. Even in mathematics, the greatest parallelogram is formed if the lines composing it are set at right angles. \textsc{Ed [Equinox]}.]

41. \textit{Conclusions from the foregoing.} It may then be suggested to the Philosophus, that although his work will be harder his reward will be greater if he choose a Deity most remote from his own nature. This method is harder and higher than that of Liber E. For a simple object as there suggested is of the same nature as the commonest things of life, while even the meanest Deity is beyond uninitiated human understanding. On the same plane, too, Venus is nearer to man than Aphrodite, Aphrodite than Isis, Isis than Babalon, Babalon than Nuit.

Let him decide therefore according to his discretion on the one hand and his aspiration on the other; and let not one outrun his fellow.

42. \textit{Further concerning the value of this Method.} Certain objections arise. Firstly, in the nature of all human love is illusion, and a certain blindness. Nor is there any true love below the Veil of the Abyss. For this reason we give this method to the Philosophus, as the reflection of the Exempt Adept, who reflects the Magister Templi and the Magus. Let then the Philosophus attain this Method as a foundation of the higher Methods to be given to him when he attains those higher grades.

Another objection lies in the partiality of this Method. This is equally a defect characteristic of the Grade.

43. \textit{Concerning a notable danger of Success.} It may occur that owing to the tremendous power of the Samadhi, overcoming all other memories as it should and does do, that the mind of the devotee may be obsessed, so that he declare his particular Deity to be sole God and Lord. This error has been the foundation of all dogmatic religions, and so the cause of more misery than all other errors combined.

The Philosophus is peculiarly liable to this because from the nature of the Method he cannot remain sceptical; he must for the time believe in his particular Deity. But let him (1) consider that this belief is only a weapon in his hands, (2) affirm sufficiently that his Deity is but an emanation or reflection or eidolon of a Being beyond him, as was said in Paragraph 2. For if he fail herein, since man cannot remain permanently in Samadhi, the memorised Image in his mind will be degraded, and replaced by the corresponding Demon, to his utter ruin.

Therefore, after Success, let him not delight overmuch in his Deity, but rather busy himself with his other work, not permitting that which is but a step to become a goal. As it is written also in Liber CLXXXV [Liber Collegii Sancti]: \enquote{remembering that Philosophy is the Equilibrium of him that is in the House of Love.}

44. \textit{Concerning the secrecy and the rites of blood.} During this practice it is most wise that the Philosophus utter no word concerning his working, as if it were a Forbidden Love that consumeth him. But let him answer fools according to their folly; for since he cannot conceal his love from his fellows, he must speak to them as they may understand.

And as many Deities demand sacrifice, one of men, another of cattle, a third of doves, let these sacrifices be replaced by the true sacrifices in thine own heart. Yet if thou must symbolise them outwardly for the hardness of thine heart, let thine own blood, and not another's, be spilt before that altar.\footnote{The exceptions to this rule pertain neither to this practice, not to this grade. N. Fra. \Argentium{}}

Nevertheless, forget not that this practice is dangerous, and may cause the manifestation of evil things, hostile and malicious, to thy great hurt.

45. \textit{Concerning a further sacrifice.} Of this it shall be understood that nothing is to be spoken; nor need anything be spoken to him that hath wisdom to comprehend the number of the paragraph. And this sacrifice is fatal beyond all, unless it be a sacrifice indeed. Yet there are those who have dared and achieved thereby.

46. \textit{Concerning yet a further sacrifice.} Here it is spoken of actual mutilation. Such acts are abominable; and while they may bring success in this Method, form an absolute bar to all further progress.

And they are in any case more likely to lead to madness than to Samadhi. He indeed who purposeth them is already mad.

47. \textit{Concerning human affection.} During this practice thou shalt in no wise withdraw thyself from human relations, only figuring to thyself that thy father or thy brother or thy wife is as it were an image of thy particular Deity. Thus shall they gain, and not lose, by thy working. Only in the case of thy wife this is difficult, since she is more to thee than all others, and in this case thou mayst act with temperance, lest her personality overcome and destroy that of thy Deity.

48. \textit{Concerning the Holy Guardian Angel.} Do thou in no wise confuse this invocation with that.

49. \textit{The Benediction.} And so may the love that passeth all Understanding keep your hearts and minds through \textgreek{\textsc{Ιαω Αδοναι Σαβαω}}, through \textsc{Babalon} of the City of the Pyramids, and through Astart\'{e}, the Starry One green-girdled, in the name \textsc{Ararita}. Amen.

