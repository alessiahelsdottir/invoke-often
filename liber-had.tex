\addchap{Liber Had}
\chapnum{DLV\footnote{Equinox Vol. 1, No. 7}}


\textbf{An instruction for attaining Hadit.}

000. This is the Book of the Cult of the Infinite Within.

00. The Aspirant is Nuit. Nuit is the infinite expansion of the Rose; Hadit the infinite concentration of the Rood. \textit{(Instruction of V.V.V.V.V.)\editorsnote{\textit{Vi veri veniversum vivus vici} --- Aleister Crowley's \grade{8}{3} motto, Latin for \enquote{By the power of truth, I, while living, have conquered the universe.}}}

0. First let the Aspirant learn in his heart the Second Chapter of the Book of the Law. \textit{(Instruction of V.V.V.V.V.)}

1. Worship, \textit{i.e.} identify thyself with, Nuit, as a lambent flame of blue, all-touching, all-penetrant, her lovely hands upon the black earth, and her lithe body arched for love, and her soft feet not hurting the little flowers, even as She is imaged in the St\'{e}l\'{e} of Revealing.

\textit{This is the first practice of Meditation (ccxx. I. 26).}

2. Let him further identify himself with the heart of Nuit, whose ecstasy is in that of her children, and her joy to see their joy, who sayeth: I love you! I yearn to you. Pale or purple, veiled or voluptuous, I who am all pleasure and purple, and drunkenness of the innermost sense, desire you. Put on the wings, and arose the coiled splendour within you: come unto me! . . . Sing the rapturous love-song unto me! Burn to me perfumes! Wear to me jewels! Drink to me, for I love you! I love you! I am the blue-lidded daughter of Sunset; I am the naked brilliance of the voluptuous night-sky. To me! To me!

\textit{This is the second practice of Meditation (ccxx. I. 13, 61, 63, 64, 65).}

3. Let the Aspirant apply himself to comprehend Hadit as an unextended point clothed with Light ineffable. And let him beware lest he be dazzled by that Light.

\textit{This is the first practice of Intelligence (ccxx. II. 2).}

4. Let the Aspirant apply himself to comprehend Hadit as the ubiquitous centre of every sphere conceivable.

\textit{This is the second practice of Intelligence (ccxx. II. 3).}

5. Let the Aspirant apply himself to comprehend Hadit as the soul of every man, and of every star, conjoining this in his Understanding with the Word (ccxx. I. 2). \enquote{Every man and every woman is a star.} Let this conception be that of Life, the giver of Life, and let him perceive that therefore the knowledge of Hadit is the knowledge of death.

\textit{This is the third practice of Intelligence (ccxx. II. 6).}

6. Let the Aspirant apply himself to comprehend Hadit as the Magician or maker of Illusion, and the Exorcist or destroyer of Illusion, under the figure of the axle of the Wheel, and the cube in the circle. Also as the Universal Soul of Motion.

(This conception harmonises Thoth and Harpocrates in a very complete and miraculous manner. Thoth is both the Magus of Taro (see Lib. 418) and the Universal Mercury; Harpocrates both the destroyer of Typhon and the Babe on the Lotus. Note that the \enquote{Ibis position} formulates this conception most exactly. \textsc{Ed [Equinox]}.)

\textit{This is the fourth practice of Intelligence (ccxx. II. 7).}

7. Let the Aspirant apply himself to comprehend Hadit as the perfect, that is Not, and solve the mystery of the numbers of Hadit and his components by his right Ingenium.

\textit{This is the fifth practice of Intelligence (ccxx. II. 15, 16).}

8. Let the Aspirant, bearing him as a great King, root out and destroy without pity all things in himself and his surroundings which are weak, dirty, or diseased, or otherwise unworthy. And let him be exceeding proud and joyous.

\textit{This is the first practice of Ethics (ccxx. II. 18, 19, 20, 21).}

9. Let the Aspirant apply himself to comprehend Hadit as the Snake that giveth Knowledge and Delight and bright glory, who stirreth the hearts of men with drunkenness. This snake is blue and gold; its eyes are red, and its spangles green and ultra-violet.

(That is, as the most exalted form of the Serpent Kundalini.)

\textit{This is the sixth practice of Intelligence (ccxx. II. 22, 50, 51).}

10. Let him further identify himself with this Snake.

\textit{This is the second practice of Meditation (ccxx. II. 22).}

11. Let the Aspirant take wine and strange drugs, according to his knowledge and experience, and be drunk thereof.

(The Aspirant should be in so sensitive a condition that a single drop, perhaps even the smell, should suffice. \textsc{Ed [Equinox]}.)

\textit{This is the first practice of Magick Art (ccxx. II. 22).}

12. Let the Aspirant concentrate his consciousness in the Rood Cross set up upon the Mountain, and identify himself with It. Let him be well aware of the difference between Its own soul, and that thought which it habitually awakes in his own mind.

\textit{This is the third practice of Meditation, and as it will be found, a comprehension and harmony and absorption of the practices of Intelligence (ccxx, II. 22).}

13. Let the Aspirant apply himself to comprehend Hadit as the Unity which is the Negative. (Ain Elohim. \textsc{Ed [Equinox]})

\textit{This is the seventh practice of Intelligence (ccxx. II. 23).}

14. Let the Aspirant live the life of a strong and beautiful being, proud and exalted, contemptuous of and fierce toward all that is base and vile.

\textit{This is the second practice of Ethics (ccxx. II. 24, 25, 45-49, 52, 56-60).}

15. Let the Aspirant apply himself to comprehend Hadit according to this 26\textsuperscript{th} verse of the Second Chapter of the Book of the Law. And this shall be easy for him if he have well accomplished the Third Practice of Meditation.

\textit{This is the eighth practice of Intelligence (ccxx, II. 26).}

16. Let the Aspirant destroy Reason in himself according to the practice in Liber CDLXXIV.

\textit{This is the fourth practice of Meditation (ccxx. II. 27-33).}

17.
Let the Aspirant observe duly the Feasts appointed by the \Argentium{} and perform such rituals of the elements as he possesseth, invoking them duly in their season.

\textit{This is the second practice of Magick Art (ccxx. II. 35-43).}

18. Let the Aspirant apply himself to comprehend Hadit as a babe in the egg of the Spirit (Akasha. \textsc{Ed [Equinox]}) that is invisible within the 4 elements.

\textit{This is the ninth practice of Intelligence (ccxx. II. 49).}

19. The Aspirant seated in his Asana will suddenly commence to breathe strangely, and this without the Operation of his will; the Inspiration will be associated with the thought of intense excitement and pleasure, even to exhaustion; and the Expiration very rapid and forceful, as if this excitement were suddenly released.

\textit{This is the first and last Indication of the Sign of the Beginning of this Result (ccxx. II. 63).}

20. A light will appear to the Aspirant, unexpectedly. Hadit will arise within him, and Nuit concentrate Herself upon him from without. He will be overcome, and the Conjunction of the Infinite Without with the Infinite Within will take place in his soul, and the One be resolved into the None.

\textit{This is the first Indication of the Nature of the Result (ccxx. II. 61, 62, 64).}

21. Let the Aspirant strengthen his body by all means in his power, and let him with equal pace refine all that is in him to the true ideal of Royalty. Yet let his formula, as a King's ought, be Excess.

\textit{This is the third practice of Ethics (ccxx. II. 70, 71).}

22. To the Aspirant who succeeds in this practice the result goes on increasing until its climax in his physical death in its due season. This practice should, however, prolong life.

\textit{This is the second Indication of the Nature of the Result (ccxx. II. 66, 72-74).}

23. Let the Adept aspire to the practice of Liber XI. and preach to mankind.

\textit{This is the fourth Practice of Ethics (ccxx. II. 76).}

24. Let the Adept worship the Name, foursquare, mystic, wonderful, of the Beast, and the name of His house; and give blessing and worship to the prophet of the lovely Star.

\textit{This is the fifth practice of Ethics (ccxx. II. 78, 79).}

25. Let the Aspirant expand his consciousness to that of Nuit, and bring it rushing inward. It may be practiced by imagining that the Heavens are falling, and then transferring the consciousness to them.

\textit{This is the fifth practice of Meditation. (Instruction of V.V.V.V.V.)}

26. Summary. Preliminaries. These are the necessary possessions.
\begin{enumerate}[leftmargin=4\parindent]
\item Wine and strange drugs.
\end{enumerate}

27. Summary continued. Preliminaries. These are the necessary comprehensions.

\begin{enumerate}[leftmargin=4\parindent]
\item The nature of Hadit (and of Nuit, and the relations between them.)
\end{enumerate}

28. Summary continued. Preliminaries. These are the meditations necessary to be accomplished.

\begin{enumerate}[leftmargin=4\parindent]

\item Identification with Nuit, body and spirit.
\item Identification with Hadit as the Snake.
\item Identification with Hadit as the Rood Cross.
\item Destruction of Reason.
\item The falling of the Heavens.
\end{enumerate}

29. Summary continued. Preliminaries. These are the Ethical Practices to be accomplished.
\begin{enumerate}[leftmargin=4\parindent]
\item The destruction of all unworthiness in one's self and one's surroundings.
\item Fulness, almost violence, of life.
\end{enumerate}

30. Summary continued. Preliminaries. These are the Magick Arts to be practiced.
\begin{enumerate}[leftmargin=4\parindent]
\item During the preparation, perform the Invocations of the Elements.
\item Observe the Feasts appointed by the \Argentium{}
\end{enumerate}

31. Summary continued. The actual Practice.

\begin{enumerate}[leftmargin=4\parindent]
\item Procure the suitable intoxication.
\item As Nuit, contract thyself with infinite force upon Hadit.
\end{enumerate}

32. Summary continued. The Results.
\begin{enumerate}[leftmargin=4\parindent]
\item Peculiar automatic breathing begins.
\item A light appears.
\item Samadhi of the two Infinites within aspirant.
\item Intensification of 3 on repetition.
\item Prolongation of life.
\item Death becomes the climax of the practice.
\end{enumerate}

33. Summary concluded. These are the practices to be performed in token of Thanksgiving for success.
\begin{enumerate}[leftmargin=4\parindent]
\item Aspiration to Liber XI.
\item Preaching of \textgreek{Θελημα} to mankind.
\item Blessing and Worship to the prophet of the lovely Star.
\end{enumerate}
